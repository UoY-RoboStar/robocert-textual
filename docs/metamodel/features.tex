% !TEX root=../robocert.tex
\newcommand{\insp}[1]{\textbf{#1}}

\begin{table}[tb]
  \caption{Comparison of \langname{} concepts with other notations.}
  \label{tab:metamodel-comparison}
  \centering

  \begin{tabular}{p{8em}p{8em}p{8em}p{16em}}
    \toprule
    \thead{\langname}
    & \thead{UML2}
    & \thead{PSC}
    & \thead{Comments}
    \\

    \midrule
    \multicolumn{4}{l}{\tsubhead{Top~(\ref{sec:metamodel-top})}}
    \\

    \midrule
    \multicolumn{4}{l}{\tsubhead{Sequences~(\ref{sec:metamodel-sequences})}}
    \\

    \msequencegap
    & n/a
    & Strict operator, \insp{constraints}
    &
      Gaps resemble past-unwanted-message constraints, but
      allow restricting the set of \emph{allowed} messages;
      this subsumes the strict operator.  Graphical syntax is a slight
      modification of PSC syntax.  We do not cover
      future-unwanted-message or chain constraints.  Currently, all
      ordering is assumed strict unless modified by a gap; this
      deviates from PSC \emph{and} some readings of UML.
    \\
    
    \midrule
    \multicolumn{4}{l}{\tsubhead{Actions~(\ref{sec:metamodel-actions})}}
    \\

    \marrowaction
    & \insp{message occurrence specification}
    & regular message
    & Arrows are named for the PSC \emph{arrowMSG} concept but are closer
      to UML.
    \\

    \mloopaction
    & \insp{loop combined fragment}
    & loop operator
    & \mloopbound s inspired by those permitted by UML.
    \\
      
    \midrule
    \multicolumn{4}{l}{\tsubhead{Messages~(\ref{sec:metamodel-messages})}}
    \\

    \mmessageset
    & n/a
    & \insp{constraint set}
    & \mrefmessageset s are directly inspired by the PSC approach to referencing constraint sets.
    \\

    \marrowmessagespec
    & ??
    & arrowMSG
    & We do not cover PSC required or fail messages.
    \\

    \mgapmessagespec
    & ??
    & \insp{intraMSG}
    & Notion of direction (inbound/outbound) rather than sender/receiver labels.
    \\

    \midrule
    \multicolumn{4}{l}{\tsubhead{Actors~(\ref{sec:metamodel-actors})}}
    \\

    \mactor
    & lifeline
    & \insp{component instance}
    &
      Always fixed at two in \langname: a \mtarget{} and a \mworld{}.
      Neither are named, though the \mtarget{} will refer to a
      named component such as a \mrcmodule.

      Like PSC, but unlike UML, there are no executions.
    \\

    \midrule
    \multicolumn{4}{l}{\tsubhead{Assertions~(\ref{sec:metamodel-assertions})}}
    \\

    \bottomrule
  \end{tabular}
\end{table}

\Cref{tab:metamodel-comparison} compares \langname{} concepts to the
corresponding features, if any, in:

\begin{itemize}
\item
  UML 2.0 sequence diagrams (including the MARTE profile for timing
  constraints);
\item
  Property Sequence Chart~\cite{psc};
\item
  \todo{more}
\end{itemize}

For each concept, we highlight the main source of inspiration (if any)
using \insp{this formatting}.

%%% Local Variables:
%%% mode: latex
%%% TeX-master: "../robocert"
%%% End:
