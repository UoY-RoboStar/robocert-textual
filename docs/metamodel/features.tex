% !TEX root=../robocert.tex
\newcommand{\insp}[1]{\textbf{#1}}

This section compares \langname{} concepts to the
corresponding features, if any, in:

\begin{itemize}
\item
  UML 2.0 sequence diagrams (including the MARTE profile for timing
  constraints);
\item
  Property Sequence Chart~\cite{psc};
\item
  \todo{more}
\end{itemize}

For each concept, we highlight the main source of inspiration (if any)
using \insp{this formatting}.

\subsection{Top~(\ref{sec:metamodel-top})}

\subsection{Sequences~(\ref{sec:metamodel-sequences})}

\paragraph{\msequencegap}
\begin{description}
\item[UML] n/a
\item[PSC] strict operator, \insp{constraints}
\item[Comments]
  Gaps resemble past-unwanted-message constraints, but
  allow restricting the set of \emph{allowed} messages;
  this subsumes the strict operator.  Graphical syntax is a slight
  modification of PSC syntax.  We do not cover
  future-unwanted-message or chain constraints.  Currently, all
  ordering is assumed strict unless modified by a gap; this
  deviates from PSC \emph{and} some readings of UML.
\end{description}
    
\subsection{Actions~(\ref{sec:metamodel-actions})}

\paragraph{\marrowaction}
\begin{description}
\item[UML] \insp{message occurrence specification}
\item[PSC] regular message
\item[Comments]
  Arrows are named for the PSC \emph{arrowMSG} concept but are closer
  to UML.
\end{description}

\paragraph{\mdeadlinestep}
\begin{description}
\item[UML] \insp{MARTE time annotations}
\item[PSC] n/a
\item[Comments]
  Syntax currently has an explicitly named initial time point, but UML2
  seems to have a similar syntax that does not.  \todo{write this}
\end{description}
      
\paragraph{\mloopstep}
\begin{description}
\item[UML] \insp{loop combined fragment}
\item[PSC] loop operator
    \mloopbound s inspired by those permitted by UML.
\end{description}
      
\subsection{Messages~(\ref{sec:metamodel-messages})}

\paragraph{\mmessageset}
\begin{description}
\item[UML] n/a
\item[PSC] \insp{constraint set}
\item[Comments]
  \mrefmessageset s are directly inspired by the PSC approach to referencing constraint sets.
\end{description}

\paragraph{\marrowmessagespec}
\begin{description}
\item[UML] ??
\item[PSC] arrowMSG
\item[Comments]
  We do not cover PSC required or fail messages.
\end{description}

\paragraph{\mgapmessagespec}
\begin{description}
\item[UML] ??
\item[PSC] \insp{intraMSG}
\item[Comments]
  Notion of direction (inbound/outbound) rather than sender/receiver labels.
\end{description}

\subsection{Actors~(\ref{sec:metamodel-actors})}

\paragraph{\mactor}
\begin{description}
\item[UML] lifeline
\item[PSC] \insp{component instance}
\item[Comments]
  Always fixed at two in \langname: a \mtarget{} and a \mworld{}.
  Neither are named, though the \mtarget{} will refer to a
  named component such as a \mrcmodule.

  Like PSC, but unlike UML, there are no executions.
\end{description}

\subsection{Assertions~(\ref{sec:metamodel-assertions})}

%%% Local Variables:
%%% mode: latex
%%% TeX-master: "../robocert"
%%% End:
