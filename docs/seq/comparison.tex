%!TEX root=../robocert.tex

This chapter compares \langname{} sequences to the following graphical
and textual notations:

\newlist{featset}{description}{1}
\setlist[featset]{font=\sffamily,align=right,labelwidth=4em}

\paragraph{Graphical (UML based)}

\begin{featset}
\item[UML] UML 2.5.1 sequence diagrams~\cite{uml251};
\item[MARTE] the MARTE profile of UML;
\item[PSC] Property Sequence Chart~\cite{psc};
\item[TPSC] Timed Property Sequence Chart~\cite{tpsc};
\item[PTPSC] Probabilistic Timed Property Sequence Chart~\cite{ptpsc}.
\end{featset}

\paragraph{Textual (Property Specification Patterns based)}

\begin{featset}
\item[PSP] Property Specification Patterns~\cite{psp,pspsite};
\item[AGLPT] Grammar given in Autili, Grunske, Lumpe, Pelliccione, and Tang~\cite{aglpt}.
\end{featset}

\paragraph{Low-level languages}
We also loosely compare \langname{} sequences to the languages targeted by their
semantics (eg, CSP).
This demonstrates the ability of our notations to
capture patterns and operators that would otherwise require using
those notations directly.

\section{Notation review}\label{sec:seq-comparison-review}
% !TEX root=../../robocert.tex

This section reviews each of the above notations in turn, comparing
\langname{} sequences to each.  (We review semantics
later on, in \cref{sec:semantics-comparison-review-seq}.)

\subsection{UML 2.5.1 sequence diagrams}

UML sequence diagrams are a well-understood, standardised notation for
specifying properties over sequences of multi-actor interactions.  As such, they
are an obvious first point of inspiration and comparison.  We track the most
recent standard at time of writing (UML 2.5.1).

\paragraph{Control flow features}
UML 2.5.1 contains generalised notions of selection (the `alt' combined fragment),
optionality (`opt'), iteration (`loop', with support for unbounded and many
forms of numerically bounded loops), parallel composition (`par'), weak ordering
(by default or with `seq'), strict ordering (`strict'), and the inclusion of
other sequences into a sequence (`ref'), amongst others.

\paragraph{Timing features}
UML 2.5.1 dedicates section 8.4 to basic notions of time.  This
support comes in \emph{time} and \emph{duration} (difference between two points
in time) primitives, further divided into \emph{observations} (inspecting the
moment in time when, or duration since, some event occurred) as well as
\emph{constraints} (specifying the time or duration of one or more events).
Constraints can to refer to observations, allowing complex situations such as
`this set of events should happen in at most three times the duration that
this other set of events happened in'.  Duration constraints take the
form of intervals between minimum and maximum permitted duration.

At the moment, our \mdeadlinestep{} construct resembles a UML duration
constraint with minimum duration \(0\).  We do not support observations.

\paragraph{Probabilistic features}
None of which we are aware, outside of extensions to STAIRS (see below).

\subsection{MARTE}

MARTE is a UML profile for real-time systems.

\paragraph{Control flow features}
As UML.

\paragraph{Timing features}
MARTE 1.2 dedicates section 9 to three classes of time abstraction:
casual, synchronous (discrete, clocked), and physical (real-time).  Our work
only covers discrete-time
situations.

MARTE has rich support, using its \emph{Value Specification Language}, for
constraining the durations of message-passing as well as parts of lifelines,
allowing constraints at both local
and global scope\todo{Getting this from Ebeid et al, but I'm not sure that's the
right citation}.  We do not yet support this level of richness.

\paragraph{Probabilistic features}
None of which we are aware.

\subsection{STAIRS}

STAIRS is an approach for incremental, refinement-based development of
UML sequence diagrams capturing existing systems.  We consider it
because it adds features that are not available in UML2, but useful when
following such development processes.

The STAIRS sense of refinement primarily involves the growth of
specifications from sequence diagrams.  It encompasses
notions of \emph{supplementing}, \emph{narrowing}, and
\emph{detailing} sequence diagrams to achieve a more
comprehensive specification.

\paragraph{Control flow features}

STAIRS adds to UML 
a distinction between \emph{potential} behaviour (nondeterminism
in the specification) and \emph{mandatory} behaviour (things the
specification \emph{must} offer the environment)---for example,
an \texttt{xalt} operator that states that all of the choices
must be offered to the environment (unlike \texttt{alt}) which
is cast as an underspecification.
This is
useful from a stepwise refinement perspective, and also
parallels the distinction in CSP between
\(\intchoice\) and \(\extchoice\).

\paragraph{Timed features}
Timed STAIRS is comparable to UML in this regard.

\paragraph{Probabilistic features}
Probabilistic STAIRS adds a
\texttt{palt} operator where choices have associated probability
sets.

\subsection{Message Sequence Chart (MSC)}

ITU message sequence charts~\cite{Harel03-MSC} are a form of sequence diagram
that serves as a precursor to many of the other notations
discussed here, including version 2 and onward of UML.

\paragraph{High-level MSC}

While MSCs do not directly support concepts such as choice and iteration
(to the best of our knowledge), there are extensions that do.  One extension
of interest is \emph{high-level} message sequence charts, which are finite state
automata whose states are labelled with message sequence charts.  This concept
leads to high levels of expressivity (while still being possible to capture in
terms of a semantics), but means that the overall diagram is no longer a single
chart.

\paragraph{Control flow features}
MSCs have very little inherent control flow, being primarily just
sequences of messages (possibly with timing constraints)

\paragraph{Timing features}
As of MSC-96, MSCs had time constraints in the form of both interval
constraints and timer/timeout pairs.

\paragraph{Probabilistic features}
There are no innate probabilistic features in MSCs.

\textcite{Krivanek09-PMSC}, in his bachelor dissertation, proposes a
probabilistic extension of MSC that adds probability in two
ways: first, in attaching probability \emph{distributions} to
durations of messages; second, in the transitions of a HMSC.  While
this proposal seems not to have been realised, the same
dissertation mentions stochastic message sequence charts (which
achieve the first point) and translation into deterministic and
stochastic Petri nets (which achieve part of the second point).

\subsection{Live Sequence Chart (LSC)}

LSC~\cite{lsc} is a
variant of MSC addressing a perceived lack of expressivity.
LSC has several features that are of interest from a \langname{} perspective:

\begin{itemize}
\item visual distinction between mandatory and provisional (or `hot'
  and `cold') elements, the metaphor being that `hot' areas of the
  chart are too hot to stay in for any period of time;
\item conditions that may also be `hot' (abort the
  chart) or `cold' (exit control flow);
\item charts directly carrying universal or existential semantics (with the
  goal of allowing a transition from use-case analysis to specifications);
\item time constraints using the setting, resetting, and comparing of timers;
\item an explicit environment lifeline, to which constraints on the
  environment can be added;
\item `pre-charts', charts that specify the previous communication
  that needs to occur for the main chart to become live.
\end{itemize}

A theme in the LSC development is progressive enhancement: charts move
from existential to universal as more use-case information appears,
and parts of charts move from cold to hot as more knowledge about
liveness appears.

While the \emph{assert} and \emph{negate} aspects of UML2 sequence diagrams
are similar to LSC temperature, \textcite{Harel08-ModalSD} argue
that temperature is not compatible with the trace-set-pair semantics typically
given to UML.  They instead propose a profile of UML implementing the
dichotomy as a modality, with a formal semantics in terms of B\"uchi automata.

LSC variants are still in use as of 2021.  For instance,
\textcite{Chai21-PMLSC} report on a use of `parameterized modal
live sequence charts' to verify train control systems at run-time.

\paragraph{Control flow features}

While the original formulation of LSC~\cite{lsc} seems not to have any
control flow features other than those available in ordinary MSC,
\textcite{Harel03-MSC} attribute both bounded/unbounded loops and some
form of alternative blocks to LSC.

\paragraph{Timing features}
While the original formulation of LSC~\cite{lsc} had no timing
features, later iterations do.
\textcite{Harel03-MSC} propose an encoding of timing constraints in
LSC using a straightforward extension of their existing notation for
constraints, using an implicit timer variable.  A later account by
\textcite{Brill04-LSCintro} suggests a different encoding of
constraints closer to MSC-96.

\paragraph{Probabilistic features}
\textcite{Kai14-PLSC} propose an extension to LSC to capture
probabilistic choice (`probabilistic switch'), as well as
probabilities over aspects of the sending, transmitting, and receiving
of events.

\subsection{Property Sequence Chart (PSC)}

PSC extends a subset of UML2 sequence diagrams (with inspiration from
Message Sequence Charts) to provide a
user-friendly layer atop linear temporal logic.  This is
similar to the general concept of \langname{} sequences with respect to \tockcsp{}
etc., though without the
specific focus on properties of \robochart{} models.
Extensions add timing (TPSC~\cite{tpsc}) and
probabilistic (PTPSC~\cite{ptpsc}) features.

A key difference between PSC and our work is that we are
targeting \todo{for now} properties easily expressed as refinement questions
(and so as artefacts such as CSP processes), whereas PSC targets properties
expressible as linear temporal logic.  We do not yet target LTL,
either directly or by any work that encodes fragments of LTL into
CSP~\cite{fdrspin,Lowe08-CommunicatingProcessSpecification}.

\paragraph{Control flow features}
PSC is considerably streamlined from UML and HMSC, with a small set of operators: strict
ordering; looping; alternative; parallel composition; messages; and constraints.
UML assertion and negation blocks do not exist, in favour of message modalities.

\paragraph{Timing features}
TPSC and PTPSC have clock constraints.  These complement the existing PSC constraint system
(which loosely corresponds to our concept of gaps), adding the ability to
specify that the occurrence of certain messages satisfies an inequality against
a named and resettable clock.

\paragraph{Probabilistic features}
PTPSC divides sequences into two sections: a LSC-style pre-chart where no messages can be required,
and a main chart with a probability inequality.  This gives a coarse-grained notion of `if we see
this behaviour, then this behaviour occurs with a certain probability'.

\subsection{Non-graphical languages}

We also consider the following languages and projects that, while not
graphical, are important in the area of property specification.

\paragraph{Property Specification Patterns (PSP)}

This long-running work by Dwyer et al. forms a repository of
structured patterns for structuring the specification of temporal
properties for concurrent and reactive systems.  These patterns have names,
known modes of composition, and example mappings into temporal logics.

\paragraph{Autili, Grunske, Lumpe, Pelliccione, and Tang (AGLPT)}

This work proposes a structured English grammar covering many of the Property
Specification Patterns.  The grammar provides for timing and probabilistic
features.

\subsection{Common themes and differences}

Summarising the above, we can find the following common
themes and differences in existing sequence
diagram languages:

\paragraph{Themes}

\begin{itemize}
\item
  Two axes, usually discrete:
  \begin{itemize}
  \item
    one distinguishing between participants in an interaction;
  \item
    another providing some form of time ordering;
  \end{itemize}
\item
  Message passing across the participant axis as the main expression of
  interaction;
\item
  Control flow manipulated by blocks 
  spanning both axes;
\item
  Often, \emph{some} notion of mandatory and provisional
  (sometimes also forbidden) behaviour;
\item
  Time and duration constraints, whether natively or through extension;
\item
  Support for probabilities, whether natively or through extension.
\end{itemize}

\paragraph{Differences}

\begin{itemize}
\item The target semantics (see
  \cref{sec:semantics-comparison-review});
\item Whether time constraints are available and, if so, how they work
  (for instance, clocks versus duration spans);
\item Whether probabilistic operators are available and, if so, how
  they work and their granularity;
\item The specifics of event ordering, especially with respect to
  communications happening on vertically disjoint lifelines but
  horizontally similar time-slices;
\item The precise form of the mandatory and provisional behaviour
  specification; for instance, temperature in LSC vs message types in
  PSC vs control blocks.
\end{itemize}

%%% Local Variables:
%%% mode: latex
%%% TeX-master: "../../robocert"
%%% End:

\section{Feature comparison}\label{sec:seq-comparison-features}
% !TEX root=../../robocert.tex
\newcommand{\insp}[1]{\ul{#1}}

This section compares \langname{} sequence features to any
corresponding features in the other notations.
For each feature, we highlight the main source of inspiration (if any)
using \insp{underlines}.

%\subsection{Sequences~(\ref{sec:metamodel-sequences})}

\subsection{Steps~(\ref{sec:metamodel-steps})}

\paragraph{\mdeadlinestep}
\begin{featset}
\item[UML] \insp{duration constraint}, time observations and constraints
\item[TPSC] clock constraint
\item[AGLPT] upper time bounds
\end{featset}

UML2 allows time and duration constraints to relate to named observations taken
previously in the sequence.  By combining a time observation at one end of a
subsequence with a time constraint at the other, we can capture a form of clock
constraint.  We do not yet capture observations.

While \robochart{} has clocks, we do not yet expose them in
\langname, as we consider them to be an implementation concern.
      
\paragraph{\mloopstep}
\begin{featset}
\item[UML] \insp{loop combined fragment}
\item[PSC] loop operator
\end{featset}

\mloopbound s are inspired by those permitted by UML.

\paragraph{Gaps}
\begin{featset}
\item[PSC] \insp{constraints}, strict operator
\item[AGLPT] until
\end{featset}

Gaps resemble past-unwanted-message constraints, but
allow restricting the set of \emph{allowed} messages;
this subsumes the strict operator.  Graphical syntax is a slight
modification of PSC syntax.  We do not cover
future-unwanted-message or chain constraints.  Currently, all
ordering is assumed strict unless modified by a gap; this
deviates from PSC \emph{and} some readings of UML.
    
\subsection{Actions~(\ref{sec:metamodel-actions})}

\paragraph{\marrowaction}
\begin{featset}
\item[UML] \insp{message occurrence specification}
\item[PSC] regular message
\end{featset}

Arrows are named for the PSC \emph{arrowMSG} concept but are closer
to UML.
      
\paragraph{\mwaitaction}
\begin{featset}
\item[UML] combination of time observations and constraints(?)
\item[TPSC] clock constraints(?)
\end{featset}

The exact form of this operator comes mostly from \robochart{} and, to
a lesser extent, \tockcsp.
Because wait actions in \langname{} assert that one
action occurs at least some time units after another, we assume that it is
possible to encode the same pattern using clock-style constraints.

\subsection{Messages~(\ref{sec:metamodel-messages})}

\paragraph{\mmessageset}
\begin{featset}
\item[PSC] \insp{constraint set}
\end{featset}

\mrefmessageset s are directly inspired by the PSC approach to referencing constraint sets.

\paragraph{\mmessagespec}
\begin{featset}
\item[UML] ??
\item[PSC] \insp{arrowMSG, intraMSG}
\end{featset}

We do not cover PSC required or fail messages.
Notion of direction (inbound/outbound) rather than sender/receiver labels.

\subsection{Actors~(\ref{sec:metamodel-actors})}

\paragraph{\mactor}
\begin{featset}
\item[UML] lifeline
\item[PSC] \insp{component instance}
\end{featset}

Unlike UML and PSC, we fix the number of actors at two
\ghtodo{32}{for now} in \langname: a \mtarget{} and a \mworld{}.
Neither are named, though the \mtarget{} will refer to a
named component such as a \mrcmodule.

Like PSC, but unlike UML, there are no executions.

\subsection{Assertions~(\ref{sec:seq-metamodel-assertions})}

As \langname{} is a notation for automated property checking,
sequence properties more closely resemble CSP-M/FDR refinement
assertions than the usual ways of reasoning about UML or PSC sequence diagrams.  As such,
features of \msequenceproperty{} capture some
concepts that are part of the diagram language 
in other notations:

\begin{itemize}
  \item
    LSC
    existential charts (specifying behaviour we expect to see at least once
    in the model) become is-observed properties in
    \langname;
  \item
    LSC notions of mandatory progress can be
    captured partially by use of tick-tock reasoning
    in \langname, though this applies to whole diagrams.
\end{itemize}

\subsection{Matrix}

\newcommand\rot{\rotatebox{90}}
\newcommand\matding[1]{{\small#1}}
\newcommand\OK{\matding{\checkmark}}
\newcommand\ASST{\matding{A}}
\newcommand\ISH{\matding{E}}
\newcommand\NO{\matding{\(\bullet\)}}
\newcommand\SOON{\matding{\(\circ\)}}
\newcommand\NA{\matding{n/a}}
\newcommand\INTIMED{\matding{T}}
\newcommand\INPROB{\matding{P}}

\begin{table}[htb!]
  \centering

  \begin{tabular}{rl|lllllll}
  \toprule

  & \rot{\thead{\langname}}
  & \rot{\thead{\featname{UML}}}
  & \rot{\thead{\featname{MARTE}}}
  & \rot{\thead{\featname{STAIRS}}}
  & \rot{\thead{\featname{LSC}}}
  & \rot{\thead{\featname{PSC}}}
  & \rot{\thead{\featname{PSP}}}
  & \rot{\thead{\featname{AGLPT}}}
  \\
  \midrule
  \multicolumn{7}{l}{\tsubhead{Modalities}}
  \\
  Universal specification
  & \OK  % Us
  & ?  % UML
  & ?  % MARTE
  & ?  % STAIRS
  & \OK  % LSC
  & ?  % PSC
  & ?  % PSP
  & ?  % AGLPT
  \\
  Existential specification
  & \ASST  % Us
  & ?  % UML
  & ?  % MARTE
  & ?  % STAIRS
  & \OK  % LSC
  & ?  % PSC
  & ?  % PSP
  & ?  % AGLPT 
  \\
  Assertion block
  & \NO  % Us
  & \OK  % UML
  & \OK  % MARTE
  & \OK  % STAIRS
  & \OK?  % LSC
  & \NO  % PSC
  & ?  % PSP
  & ?  % AGLPT 
  \\ 
  Negation block
  & \NO  % Us
  & \OK  % UML
  & \OK  % MARTE
  & \OK  % STAIRS
  & ?  % LSC
  & \NO  % PSC
  & ?  % PSP
  & ?  % AGLPT 
  \\ 
  Mandatory progress (liveness)
  & \ASST  % Us
  & ?  % UML
  & ?  % MARTE
  & ?  % STAIRS
  & \OK  % LSC
  & ?  % PSC
  & ?  % PSP
  & ?  % AGLPT 
  \\  
  \midrule
  \multicolumn{7}{l}{\tsubhead{Messages}}
  \\
  Asynchronous
  & \OK  % Us
  & \OK  % UML
  & \OK  % MARTE
  & \OK  % STAIRS
  & ?  % LSC
  & \NO  % PSC
  & ?  % PSP
  & ?  % AGLPT
  \\
  Synchronous
  & \SOON?  % Us
  & \OK  % UML
  & \OK  % MARTE
  & \OK  % STAIRS
  & ?  % LSC
  & \OK  % PSC
  & ?  % PSP
  & ?  % AGLPT
  \\
  Expected/fail
  & \NO  % Us
  & \NO  % UML
  & \NO  % MARTE
  & \NO  % STAIRS
  & \OK  % LSC
  & \OK  % PSC
  & ?  % PSP
  & ?  % AGLPT
  \\
  \midrule
  \multicolumn{7}{l}{\tsubhead{Waits}}
  \\
  Explicit (\mwaitaction)
  & \OK  % Us
  & \ISH  % UML
  & \ISH  % MARTE
  & \INTIMED  % STAIRS
  & ?  % LSC
  & \INTIMED?  % PSC
  & ?  % PSP
  & ?  % AGLPT
  \\
  Ranged
  & \SOON  % Us
  & \ISH  % UML
  & \ISH  % MARTE
  & \INTIMED  % STAIRS
  & ?  % LSC
  & \INTIMED?  % PSC
  & ?  % PSP
  & ?  % AGLPT
  \\
  \midrule
  \multicolumn{7}{l}{\tsubhead{Duration constraints}}
  \\
  Lower-bound
  & \SOON?  % Us
  & \OK  % UML
  & \OK  % MARTE
  & \INTIMED  % STAIRS
  & ?  % LSC
  & \INTIMED?  % PSC
  & ?  % PSP
  & ?  % AGLPT
  \\
  Upper-bound/deadline
  & \OK  % Us
  & \OK  % UML
  & \OK  % MARTE
  & \INTIMED  % STAIRS
  & \OK  % LSC
  & \INTIMED  % PSC
  & ?  % PSP
  & ?  % AGLPT
  \\
  \midrule
  \multicolumn{7}{l}{\tsubhead{Conditionally executed blocks}}
  \\
  Optional (\texttt{opt})
  & \SOON?  % Us
  & \OK  % UML
  & \OK  % MARTE
  & \OK  % STAIRS
  & ?  % LSC
  & \ISH  % PSC
  & ?  % PSP
  & ?  % AGLPT
  \\
  Probabilistic optional
  & \SOON  % Us
  & \NO  % UML
  & \NO  % MARTE
  & \NO  % STAIRS
  & ?  % LSC
  & \INPROB  % PSC
  & ?  % PSP
  & ?  % AGLPT
  \\
  Potential alternative (\texttt{alt}, \(\sqcap\))
  & \SOON  % Us
  & \OK  % UML
  & \OK  % MARTE
  & \OK  % STAIRS
  & ?  % LSC
  & \OK  % PSC
  & ?  % PSP
  & ?  % AGLPT
  \\
  Mandatory alternative (\texttt{xalt}, \(\extchoice\))
  & \SOON?  % Us
  & \NO  % UML
  & \NO  % MARTE
  & \OK  % STAIRS
  & ?  % LSC
  & \OK  % PSC
  & ?  % PSP
  & ?  % AGLPT
  \\
  Probabilistic alternative (\texttt{palt})
  & \SOON?  % Us
  & \NO  % UML
  & \NO  % MARTE
  & \INPROB  % STAIRS
  & ?  % LSC
  & ?  % PSC
  & ?  % PSP
  & ?  % AGLPT
  \\
  \midrule
  \multicolumn{7}{l}{\tsubhead{Other}}
  \\
  Gaps (or similar)
  & \OK  % Us
  & \NO  % UML
  & \NO  % MARTE
  & \NO  % STAIRS
  & ?  % LSC
  & \OK  % PSC
  & ?  % PSP
  & \ISH?  % AGLPT
  \\
  Loops (\mloopstep)
  & \OK  % Us
  & \OK  % UML
  & \OK  % MARTE
  & \OK  % STAIRS
  & ?  % LSC
  & \OK  % PSC
  & ?  % PSP
  & ?  % AGLPT
  \\
  Breaks
  & \SOON  % Us
  & \OK  % UML
  & \OK  % MARTE
  & \OK  % STAIRS
  & ?  % LSC
  & \OK  % PSC
  & ?  % PSP
  & ?  % AGLPT
  \\
  \bottomrule
  \end{tabular}
  \caption{Matrix of features available in \langname{} sequences compared to
  other notations.}
  \label{tab:seq-comparison-features}
\end{table}

\Cref{tab:seq-comparison-features} summarises the \langname{} features listed earlier
(which may correspond to similar features in other
notations) as well as features in other notations for which \langname{} does not yet have a counterpart.
The symbols in this table have the following meaning:

\begin{description}
  \item[\OK] Directly supported.
  \item[\SOON] Not yet supported, but
  support is planned.
  \item[\NO] Not supported or planned.
  \item[\ISH] Not directly supported, but can be encoded using other
  features.
  \item[\ASST] Not directly supported in the
  diagram language, but can be encoded using the
  assertion language.  For example, \langname{} handles
  the difference between universal and existential
  quantifications in assertions.
  \item[\INTIMED] Supported, but only in timed variants of this
  notation (for example, timed STAIRS).
  \item[\INPROB] Supported, but only in
  probabilistic variants of this
  notation (for example, probabilistic PSC). 
  \item[?] No or incomplete data.
\end{description}

%%% Local Variables:
%%% mode: latex
%%% TeX-master: "../../robocert"
%%% End:


%%% Local Variables:
%%% mode: latex
%%% TeX-master: "../robocert"
%%% End:
