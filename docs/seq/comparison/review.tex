% !TEX root=../../robocert.tex

This section reviews each of the above notations in turn, comparing
\langname{} sequences to each.  (We discuss semantics
later on, in \cref{sec:semantics-review-seq}.)

\subsection{UML 2.5.1 sequence diagrams}

UML sequence diagrams are a well-understood, standardised notation for
specifying properties over sequences of multi-actor interactions.  As such, they
are an obvious first point of inspiration and comparison.  We track the most
recent standard at time of writing (UML 2.5.1).

\paragraph{Timing features}
UML 2.5.1 dedicates section 8.4 to basic notions of time.  This
support comes in \emph{time} and \emph{duration} (difference between two points
in time) primitives, further divided into \emph{observations} (inspecting the
moment in time when, or duration since, some event occurred) as well as
\emph{constraints} (specifying the time or duration of one or more events).
Constraints can to refer to observations, allowing complex situations such as
`this set of events should happen in at most three times the duration that
this other set of events happened in'.  Duration constraints take the
form of intervals between minimum and maximum permitted duration.

At the moment, our \mdeadlinestep{} construct resembles a UML duration
constraint with minimum duration \(0\).  We do not support observations.

\paragraph{Probabilistic features}
None.  \todo{check}

\subsection{MARTE}

MARTE is a UML profile for real-time systems.

\paragraph{Timing features}
MARTE 1.2 dedicates section 9 to three classes of time abstraction:
casual, synchronous (discrete, clocked), and physical (real-time).  Our work
only covers discrete-time
situations.

MARTE has rich support, using its \emph{Value Specification Language}, for
constraining the durations of message-passing as well as parts of lifelines,
allowing constraints at both local
and global scope\todo{Getting this from Ebeid et al, but I'm not sure that's the
right citation}.  We do not yet support this level of richness.

\paragraph{Probabilistic features}
None.  \todo{check}

\subsection{STAIRS}

STAIRS is an approach for incremental, refinement-based development of
UML sequence diagrams capturing existing systems.  We compare against it
because it adds features that are not available in UML2, but useful when
following such development processes:

\begin{itemize}
\item
	a distinction between \emph{potential} behaviour (nondeterminism
	in the specification) and \emph{mandatory} behaviour (things the
	specification \emph{must} offer the environment)---for example,
	an \texttt{xalt} operator that states that all of the choices
	must be offered to the environment (unlike \texttt{alt}) which
	is cast as an underspecification.
	This is
	useful from a stepwise refinement perspective, and also
	parallels the distinction in CSP between
	\(\intchoice\) and \(\extchoice\);
\item
	in the probabilistic form of STAIRS, an
	\texttt{alt} operator where choices have associated probability
	sets;
\item
	notions of \emph{supplementing}, \emph{narrowing}, and
	\emph{detailing} sequence diagrams to achieve a more
	comprehensive specification.
\end{itemize}

\subsection{Property Sequence Chart (PSC)}

PSC extends a subset of UML2 sequence diagrams (with inspiration from
Message Sequence Charts) to provide a
user-friendly layer atop linear temporal logic.  This is
similar to the general concept of \langname{} sequences with respect to \tockcsp{}
etc., though without the
specific focus on properties of \robochart{} models.
Extensions add timing (TPSC~\cite{tpsc}) and
probabilistic (PTPSC~\cite{ptpsc}) features.

A key difference between PSC and our work is that we are
targeting \todo{for now} properties easily expressed as refinement questions
(and so as artefacts such as CSP processes), whereas PSC targets properties
expressible as linear temporal logic.  While there exist ways of encoding LTL
in CSP~\cite{fdrspin}, we do not \todo{yet} use them here.

\paragraph{Timing features}
TPSC and PTPSC have clock constraints.  These complement the existing PSC constraint system
(which loosely corresponds to our concept of gaps), adding the ability to
specify that the occurrence of certain messages satisfies an inequality against
a named and resettable clock.

\paragraph{Probabilistic features}
PTPSC add probabilistic
sections.  Such sections resemble UML combined fragments with inequalities over
the probability of entering the section.

\subsection{Property Specification Patterns (PSP)}

This long-running work by Dwyer et al. forms a repository of
structured patterns for structuring the specification of temporal
properties for concurrent and reactive systems.  These patterns have names,
known modes of composition, and example mappings into temporal logics.

\subsection{Autili, Grunske, Lumpe, Pelliccione, and Tang (AGLPT)}

This work proposes a structured English grammar covering many of the Property
Specification Patterns.  The grammar provides for timing and probabilistic
features.

%%% Local Variables:
%%% mode: latex
%%% TeX-master: "../../robocert"
%%% End: