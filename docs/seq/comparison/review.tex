% !TEX root=../../robocert.tex

This section reviews each of the above notations in turn, comparing
\langname{} sequences to each.

\subsection{UML 2.5.1 sequence diagrams}

UML sequence diagrams are a well-understood, standardised notation for
specifying properties over sequences of multi-actor interactions.  As such, they
are an obvious first point of inspiration and comparison.  We track the most
recent standard at time of writing (2.5.1) here.

\paragraph{Timing features}
UML 2.5.1 dedicates section 8.4 to basic notions of time.  This
support comes in \emph{time} and \emph{duration} (difference between two points
in time) primitives, further divided into \emph{observations} (inspecting the
moment in time when, or duration since, some event occurred) as well as
\emph{constraints} (specifying the time or duration of one or more events).
Constraints can to refer to observations, allowing complex situations such as
`this set of events should happen in at most three times the duration that
this other set of events happened in'.  Duration constraints come in the
form of intervals between minimum and maximum permitted duration.

At the moment, our \mdeadlinestep{} construct resembles a UML duration
constraint with minimum duration \(0\).  We do not support observations.

\paragraph{Probabilistic features}
None.  \todo{check}

\paragraph{Semantics}
Our semantics (\cref{sec:semantics-tockcsp-seq}) takes heavy inspiration from
that of Lima et al.~\cite{lima-semantics}, with minor differences to account
for the fact that we target \tockcsp{} and not Compass CML.  In the case of
two-\mactor{} sequences \todo{all of them at the moment}, we can (and do)
simplify the semantics significantly.
The main novelty of our work here is that, by targeting \tockcsp, we can capture
timing properties.

\subsection{MARTE}

MARTE is a UML profile for real-time systems.

\paragraph{Timing features}
MARTE 1.2 dedicates section 9 to three classes of time abstraction:
casual, synchronous (discrete, clocked), and physical (real-time).  Our work
only covers discrete-time
situations.

MARTE has rich support, using its \emph{Value Specification Language}, for
constraining the durations of message-passing as well as parts of lifelines,
allowing constraints at both local
and global scope\todo{Getting this from Ebeid et al, but I'm not sure that's the
right citation}.  We do not yet support this level of richness.

\paragraph{Probabilistic features}
None.  \todo{check}

\paragraph{Semantics}
We are not aware of a particular formal semantics for the MARTE profile.
\todo{Check}
There is work to generate HDL from MARTE sequence diagrams~\todo{cite}

\subsection{Property Sequence Chart (PSC)}

This notation extends a subset of UML2 sequence diagrams (with inspiration from
Message Sequence Charts) to provide a
user-friendly layer atop linear temporal logic.  This is
similar to the general concept of \langname{} sequences with respect to \tockcsp{}
etc., though without the
specific focus on describing properties of \robochart{} models.

A key difference between PSC and our work is that we are
targeting \todo{for now} properties easily expressed as refinement questions
(and so as artefacts such as CSP processes), whereas PSC targets properties
expressible as linear temporal logic.  While there exist ways of encoding LTL
in CSP~\cite{fdrspin}, we do not \todo{yet} use them here.

\paragraph{Timing features}
None.

\paragraph{Probabilistic features}
None.

\paragraph{Semantics}
The PSC semantics is in terms of B\"uchi automata, and automated as a
\textsc{Charmy} plugin; there also exists a denotational semantics in terms of
invalid traces.

\subsection{Timed Property Sequence Chart (TPSC)}
This notation extends property sequence charts to add clock constraints.  Many
of the observations for PSC above apply here, also.

\paragraph{Timing features}
\todo{TODO}

\paragraph{Probabilistic features}
None.

\paragraph{Semantics}
The TPSC semantics is in terms of timed B\"uchi automata.

\subsection{Probabilistic Timed Property Sequence Chart (PTPSC)}
This notation extends timed property sequence charts to add probabilistic
sections.  Such sections resemble UML combined fragments with inequalities over
the probability of entering the section.

\paragraph{Timing features}
As above.

\paragraph{Probabilistic features}
\todo{TODO}

\paragraph{Semantics}
The PTPSC semantics also appears to be in terms of timed B\"uchi automata,
augmented with monitoring processes.  \todo{I don't quite understand this yet.}

\subsection{Property Specification Patterns (PSP)}

This long-running work by Dwyer et al. forms a repository of
`Gang of Four'-style patterns for structuring the specification of temporal
properties for concurrent and reactive systems.  These patterns have names,
known modes of composition, and example mappings into temporal logics.

\subsection{Autili, Grunske, Lumpe, Pelliccione, and Tang (AGLPT)}

This work proposes a structured English grammar covering many of the Property
Specification Patterns.  The grammar provides for timing and probabilistic
features.

%%% Local Variables:
%%% mode: latex
%%% TeX-master: "../../robocert"
%%% End: