% !TEX root=../../robocert.tex

This section reviews each of the above notations in turn, comparing
\langname{} sequences to each.  (We review semantics
later on, in \cref{sec:semantics-comparison-review-seq}.)

\subsection{UML 2.5.1 sequence diagrams}

UML sequence diagrams are a well-understood, standardised notation for
specifying properties over sequences of multi-actor interactions.  As such, they
are an obvious first point of inspiration and comparison.  We track the most
recent standard at time of writing (UML 2.5.1).

\paragraph{Timing features}
UML 2.5.1 dedicates section 8.4 to basic notions of time.  This
support comes in \emph{time} and \emph{duration} (difference between two points
in time) primitives, further divided into \emph{observations} (inspecting the
moment in time when, or duration since, some event occurred) as well as
\emph{constraints} (specifying the time or duration of one or more events).
Constraints can to refer to observations, allowing complex situations such as
`this set of events should happen in at most three times the duration that
this other set of events happened in'.  Duration constraints take the
form of intervals between minimum and maximum permitted duration.

At the moment, our \mdeadlinestep{} construct resembles a UML duration
constraint with minimum duration \(0\).  We do not support observations.

\paragraph{Probabilistic features}
None.  \todo{check}

\subsection{MARTE}

MARTE is a UML profile for real-time systems.

\paragraph{Timing features}
MARTE 1.2 dedicates section 9 to three classes of time abstraction:
casual, synchronous (discrete, clocked), and physical (real-time).  Our work
only covers discrete-time
situations.

MARTE has rich support, using its \emph{Value Specification Language}, for
constraining the durations of message-passing as well as parts of lifelines,
allowing constraints at both local
and global scope\todo{Getting this from Ebeid et al, but I'm not sure that's the
right citation}.  We do not yet support this level of richness.

\paragraph{Probabilistic features}
None.  \todo{check}

\subsection{STAIRS}

STAIRS is an approach for incremental, refinement-based development of
UML sequence diagrams capturing existing systems.  We compare against it
because it adds features that are not available in UML2, but useful when
following such development processes:

\begin{itemize}
\item
	a distinction between \emph{potential} behaviour (nondeterminism
	in the specification) and \emph{mandatory} behaviour (things the
	specification \emph{must} offer the environment)---for example,
	an \texttt{xalt} operator that states that all of the choices
	must be offered to the environment (unlike \texttt{alt}) which
	is cast as an underspecification.
	This is
	useful from a stepwise refinement perspective, and also
	parallels the distinction in CSP between
	\(\intchoice\) and \(\extchoice\);
\item
	in the probabilistic form of STAIRS, an
	\texttt{alt} operator where choices have associated probability
	sets;
\item
	notions of \emph{supplementing}, \emph{narrowing}, and
	\emph{detailing} sequence diagrams to achieve a more
	comprehensive specification.
\end{itemize}

\subsection{Live Sequence Chart (LSC)}

LSC~\cite{lsc} is a
variant of ITU message sequence charts (similar to UML sequence
diagrams).
LSC has several features that are of interest from a \langname{} perspective:

\begin{itemize}
\item visual distinction between mandatory and provisional (or `hot'
  and `cold') elements, the metaphor being that `hot' areas of the
  chart are too hot to stay in for any period of time;
\item conditions that also have a dichotomy between `hot' (abort the
  chart) and `cold' (exit control flow);
\item charts directly carrying universal or existential semantics (with the
  goal of allowing a transition from use-case analysis to specifications);
\item time constraints using the setting, resetting, and comparing of timers;
\item an explicit environment lifeline, to which constraints on the
  environment can be added;
\item `pre-charts', charts that specify the previous communication
  that needs to occur for the main chart to become live.
\end{itemize}

A theme in the LSC development is progressive enhancement: charts move
from existential to universal as more use case information appears,
and parts of charts move from cold to hot as more knowledge about
liveness appears.

While the \emph{assert} and \emph{negate} aspects of UML2 sequence diagrams
are similar to LSC temperature, Harel and Maoz~\cite{Harel08-ModalSD} argue
that temperature is not compatible with the trace-set-pair semantics typically
given to UML.  They instead propose a profile of UML implementing the
dichotomy as a modality, with a formal semantics in terms of B\"uchi automata.

LSC variants are still in use today.  For instance,
Chai et al. 2021~\cite{Chai21-PMLSC} report on a use of `parameterized modal
live sequence charts' to verify train control systems at run-time.


\subsection{Property Sequence Chart (PSC)}

PSC extends a subset of UML2 sequence diagrams (with inspiration from
Message Sequence Charts) to provide a
user-friendly layer atop linear temporal logic.  This is
similar to the general concept of \langname{} sequences with respect to \tockcsp{}
etc., though without the
specific focus on properties of \robochart{} models.
Extensions add timing (TPSC~\cite{tpsc}) and
probabilistic (PTPSC~\cite{ptpsc}) features.

A key difference between PSC and our work is that we are
targeting \todo{for now} properties easily expressed as refinement questions
(and so as artefacts such as CSP processes), whereas PSC targets properties
expressible as linear temporal logic.  We do not yet target LTL,
either directly or by any work that encodes fragments of LTL into
CSP~\cite{fdrspin,Lowe08-CommunicatingProcessSpecification}.

\paragraph{Timing features}
TPSC and PTPSC have clock constraints.  These complement the existing PSC constraint system
(which loosely corresponds to our concept of gaps), adding the ability to
specify that the occurrence of certain messages satisfies an inequality against
a named and resettable clock.

\paragraph{Probabilistic features}
PTPSC add probabilistic
sections.  Such sections resemble UML combined fragments with inequalities over
the probability of entering the section.

\subsection{Non-graphical languages}

\paragraph{Property Specification Patterns (PSP)}

This long-running work by Dwyer et al. forms a repository of
structured patterns for structuring the specification of temporal
properties for concurrent and reactive systems.  These patterns have names,
known modes of composition, and example mappings into temporal logics.

\paragraph{Autili, Grunske, Lumpe, Pelliccione, and Tang (AGLPT)}

This work proposes a structured English grammar covering many of the Property
Specification Patterns.  The grammar provides for timing and probabilistic
features.

\subsection{Common themes and differences}

Summarising the above, we can find the following common
themes and differences in existing sequence
diagram languages:

\paragraph{Themes}

\begin{itemize}
\item
	Two axes, usually discrete:
	\begin{itemize}
		\item
			one distinguishing between participants in an interaction;
		\item
			another providing some form of time ordering;
	\end{itemize}
\item
	Message passing across the participant axis as the main expression of
	interaction;
\item
	Control flow manipulated by blocks 
	spanning both axes;
\item
  Often, \emph{some} notion of mandatory and provisional
  (sometimes also forbidden) behaviour.
\end{itemize}

\paragraph{Differences}

\begin{itemize}
\item
	The target semantics (see \cref{sec:semantics-comparison-review});
\item
	Whether time constraints are available and, if so, how they work
	(for instance, clocks versus duration spans);
\item
	The specifics of event ordering, especially with respect to communications
	happening on vertically disjoint lifelines but horizontally
	similar time-slices;
\item
	The precise form of the mandatory and provisional
	behaviour specification; for instance, temperature in LSC
	vs message types in PSC vs control blocks.
\end{itemize}

%%% Local Variables:
%%% mode: latex
%%% TeX-master: "../../robocert"
%%% End: