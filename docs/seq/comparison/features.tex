% !TEX root=../../robocert.tex
\newcommand{\insp}[1]{\ul{#1}}

This section compares \langname{} sequence features to any
corresponding features in the other notations.
For each feature, we highlight the main source of inspiration (if any)
using \insp{underlines}.

As well as the notations reviewed previously, we also
informally note where \langname{} sequence features align to those of
languages targeted by the \langname{} semantics.  These languages are:

\begin{featset}
\item[CSP] \tockcsp~(\autoref{cha:semantics-tockcsp}).
\end{featset}
%\subsection{Sequences~(\ref{sec:metamodel-sequences})}

\subsection{Steps~(\ref{sec:metamodel-steps})}

\paragraph{\mdeadlinestep}
\begin{featset}
\item[UML] \insp{duration constraint}, time observations and constraints
\item[TPSC] clock constraint
\item[AGLPT] upper time bounds
\item[CSP] \insp{deadline operator}
\end{featset}

UML2 allows time and duration constraints to relate to named observations taken
previously in the sequence.  By combining a time observation at one end of a
subsequence with a time constraint at the other, we can capture a form of clock
constraint.  We do not yet capture observations.

While \robochart{} has clocks, we do not yet expose them in
\langname, as we consider them to be an implementation concern.
      
\paragraph{\mloopstep}
\begin{featset}
\item[UML] \insp{loop combined fragment}
\item[PSC] loop operator
\item[CSP] recursive processes
\end{featset}

\mloopbound s are inspired by those permitted by UML.

\paragraph{Gaps}
\begin{featset}
\item[PSC] \insp{constraints}, strict operator
\item[AGLPT] until
\item[CSP] \insp{interrupt operator}, throw operator
\end{featset}

Gaps resemble past-unwanted-message constraints, but
allow restricting the set of \emph{allowed} messages;
this subsumes the strict operator.  Graphical syntax is a slight
modification of PSC syntax.  We do not cover
future-unwanted-message or chain constraints.  Currently, all
ordering is assumed strict unless modified by a gap; this
deviates from PSC \emph{and} some readings of UML.
    
\subsection{Actions~(\ref{sec:metamodel-actions})}

\paragraph{\marrowaction}
\begin{featset}
\item[UML] \insp{message occurrence specification}
\item[PSC] regular message
\item[CSP] prefix
\end{featset}

Arrows are named for the PSC \emph{arrowMSG} concept but are closer
to UML.
      
\subsection{Messages~(\ref{sec:metamodel-messages})}

\paragraph{\mmessageset}
\begin{featset}
\item[PSC] \insp{constraint set}
\end{featset}

\mrefmessageset s are directly inspired by the PSC approach to referencing constraint sets.

\paragraph{\mmessagespec}
\begin{featset}
\item[UML] ??
\item[PSC] \insp{arrowMSG, intraMSG}
\end{featset}

We do not cover PSC required or fail messages.
Notion of direction (inbound/outbound) rather than sender/receiver labels.

\subsection{Actors~(\ref{sec:metamodel-actors})}

\paragraph{\mactor}
\begin{featset}
\item[UML] lifeline
\item[PSC] \insp{component instance}
\end{featset}

Always fixed at two \ghtodo{32}{for now} in \langname: a \mtarget{} and a \mworld{}.
Neither are named, though the \mtarget{} will refer to a
named component such as a \mrcmodule.

Like PSC, but unlike UML, there are no executions.

\subsection{Matrix}

\newcommand\rot{\rotatebox{90}}
\newcommand\OK{\checkmark}
\newcommand\ISH{(\OK)}
\newcommand\NO{\(\bullet\)}
\newcommand\SOON{\(\circ\)}

\begin{table}[htb!]
  \label{tab:seq-comparison-features}
  \centering

  \begin{tabular}{rl|ccccc|cc|ccc}
  \toprule

  & \rot{\thead{\langname}}
  & \rot{\thead{UML}}
  & \rot{\thead{MARTE}}
  & \rot{\thead{PSC}}
  & \rot{\thead{TPSC}}
  & \rot{\thead{PTPSC}}
  & \rot{\thead{PSP}}
  & \rot{\thead{AGLPT}}
  & \rot{\thead{CSP}}
  & \rot{\thead{PRISM}}
  & \rot{\thead{IUTP}}
  \\
  \midrule
  \tsubhead{Regular messages} & \OK & \OK & \OK & \OK & \OK & \OK & ? & ? & \OK & ? & ?
  \\
  \tsubhead{Expected messages} & \NO & \NO & \NO & \OK & \OK & \OK & ? & ? & ? & ? & ?
  \\
  \tsubhead{Fail messages} & \NO & \NO & \NO & \OK & \OK & \OK & ? & ? & ? & ? & ?
  \\
  \midrule
  \tsubhead{Waits (\mwaitaction)} & \OK & \ISH & \ISH & \NO & \ISH? & \ISH? & ? & ? & \OK & ? & ?
  \\
  \tsubhead{Nondeterministic waits} & \SOON & \ISH & \ISH & \NO & \ISH? & \ISH? & ? & ? & \OK & ? & ?
  \\
  \midrule
  \tsubhead{Gaps/inter-msg constraints} & \OK & \NO & \NO & \OK & \OK & \OK & ? & \ISH? & \ISH & ? & ?
  \\
  \tsubhead{Deadlines (\mdeadlinestep)} & \OK & \OK & \OK & \NO & \OK & \OK & ? & ? & \OK & ? & ?

  \\
  \tsubhead{Loops (\mloopstep)} & \OK & \OK & \OK & \OK & \OK & \OK & ? & ? & \OK & ? & ?
  \\
  \tsubhead{Breaks} & \SOON & \OK & \OK & \OK & \OK & \OK & ? & ? & \ISH & ? & ?
  \\
  \tsubhead{Optional blocks} & \SOON? & \OK & \OK & \ISH & \ISH & \ISH & ? & ? & \OK & ? & ?
  \\
  \tsubhead{Probabilistic optional blocks} & \SOON & \NO & \NO & \NO & \NO & \OK & ? & ? & \NO & ? & ?
  \\
  \bottomrule
  \end{tabular}
  \caption{Matrix of features available in \langname{} sequences compared to
  other notations.  (Key: \OK{}={}supported; \ISH{}={}indirectly encodable; \SOON{}={}planned; \NO{}={}unsupported)}
\end{table}

\Cref{tab:seq-comparison-features} summarises both the features listed earlier
(which \langname{} has, and may correspond to similar features in other
notations) and features in other notations for which \langname{} does not,
or not yet, have a counterpart.

%%% Local Variables:
%%% mode: latex
%%% TeX-master: "../../robocert"
%%% End:
