% !TEX root=../../robocert.tex
\newcommand{\insp}[1]{\ul{#1}}

This section compares \langname{} sequence features to any
corresponding features in the other notations.
For each feature, we highlight the main source of inspiration (if any)
using \insp{underlines}.

\subsection{Sequences~(\ref{sec:metamodel-sequences})}

\subsection{Steps~(\ref{sec:metamodel-steps})}

\paragraph{\mdeadlinestep}
\begin{featset}
\item[UML] \insp{duration constraint}, time observations and constraints
\item[TPSC] clock constraint
\item[AGLPT] upper time bounds
\end{featset}

UML2 allows time and duration constraints to relate to named observations taken
previously in the sequence.  By combining a time observation at one end of a
subsequence with a time constraint at the other, we can capture a form of clock
constraint.  We do not yet capture observations.

While \robochart{} has clocks, we do not yet expose them in
\langname, as we consider them to be an implementation concern.
      
\paragraph{\mloopstep}
\begin{featset}
\item[UML] \insp{loop combined fragment}
\item[PSC] loop operator
\end{featset}

\mloopbound s are inspired by those permitted by UML.

\paragraph{Gaps}
\begin{featset}
\item[PSC] strict operator, \insp{constraints}
\end{featset}

Gaps resemble past-unwanted-message constraints, but
allow restricting the set of \emph{allowed} messages;
this subsumes the strict operator.  Graphical syntax is a slight
modification of PSC syntax.  We do not cover
future-unwanted-message or chain constraints.  Currently, all
ordering is assumed strict unless modified by a gap; this
deviates from PSC \emph{and} some readings of UML.
    
\subsection{Actions~(\ref{sec:metamodel-actions})}

\paragraph{\marrowaction}
\begin{featset}
\item[UML] \insp{message occurrence specification}
\item[PSC] regular message
\end{featset}

Arrows are named for the PSC \emph{arrowMSG} concept but are closer
to UML.
      
\subsection{Messages~(\ref{sec:metamodel-messages})}

\paragraph{\mmessageset}
\begin{featset}
\item[PSC] \insp{constraint set}
\end{featset}

\mrefmessageset s are directly inspired by the PSC approach to referencing constraint sets.

\paragraph{\mmessagespec}
\begin{featset}
\item[UML] ??
\item[PSC] \insp{arrowMSG, intraMSG}
\end{featset}

We do not cover PSC required or fail messages.
Notion of direction (inbound/outbound) rather than sender/receiver labels.

\subsection{Actors~(\ref{sec:metamodel-actors})}

\paragraph{\mactor}
\begin{featset}
\item[UML] lifeline
\item[PSC] \insp{component instance}
\end{featset}

Always fixed at two \todo{for now} in \langname: a \mtarget{} and a \mworld{}.
Neither are named, though the \mtarget{} will refer to a
named component such as a \mrcmodule.

Like PSC, but unlike UML, there are no executions.

%%% Local Variables:
%%% mode: latex
%%% TeX-master: "../../robocert"
%%% End:
