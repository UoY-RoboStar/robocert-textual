% !TEX root=../../robocert.tex
\newcommand{\insp}[1]{\ul{#1}}

This section compares \langname{} sequence features to any
corresponding features in the other notations.
For each feature, we highlight the main source of inspiration (if any)
using \insp{underlines}.

\paragraph{Additional points of comparison}

As well as the notations reviewed previously, we also
informally note where \langname{} sequence features align to those of
languages targeted by the \langname{} semantics.  These languages are:

\begin{featset}
\item[CSP] \tockcsp~(\autoref{cha:semantics-tockcsp}).
\end{featset}
%\subsection{Sequences~(\ref{sec:metamodel-sequences})}

We also note where features in \langname{} correspond directly to concepts
in \robochart{} \todo{and other languages}:

\begin{featset}
\item[RC] \robochart.
\end{featset}

\subsection{Steps~(\ref{sec:metamodel-steps})}

\paragraph{\mdeadlinestep}
\begin{featset}
\item[UML] \insp{duration constraint}, time observations and constraints
\item[TPSC] clock constraint
\item[AGLPT] upper time bounds
\item[CSP] \insp{deadline operator}
\item[RC] \insp{deadlines}
\end{featset}

UML2 allows time and duration constraints to relate to named observations taken
previously in the sequence.  By combining a time observation at one end of a
subsequence with a time constraint at the other, we can capture a form of clock
constraint.  We do not yet capture observations.

While \robochart{} has clocks, we do not yet expose them in
\langname, as we consider them to be an implementation concern.
      
\paragraph{\mloopstep}
\begin{featset}
\item[UML] \insp{loop combined fragment}
\item[PSC] loop operator
\item[CSP] recursive processes
\end{featset}

\mloopbound s are inspired by those permitted by UML.

\paragraph{Gaps}
\begin{featset}
\item[PSC] \insp{constraints}, strict operator
\item[AGLPT] until
\item[CSP] \insp{interrupt operator}, throw operator
\end{featset}

Gaps resemble past-unwanted-message constraints, but
allow restricting the set of \emph{allowed} messages;
this subsumes the strict operator.  Graphical syntax is a slight
modification of PSC syntax.  We do not cover
future-unwanted-message or chain constraints.  Currently, all
ordering is assumed strict unless modified by a gap; this
deviates from PSC \emph{and} some readings of UML.
    
\subsection{Actions~(\ref{sec:metamodel-actions})}

\paragraph{\marrowaction}
\begin{featset}
\item[UML] \insp{message occurrence specification}
\item[PSC] regular message
\item[CSP] prefix
\end{featset}

Arrows are named for the PSC \emph{arrowMSG} concept but are closer
to UML.
      
\paragraph{\mwaitaction}
\begin{featset}
\item[UML] combination of time observations and constraints(?)
\item[TPSC] clock constraints(?)
\item[CSP] \insp{wait}
\item[RC] \insp{wait}
\end{featset}

Because the use of a wait action in \langname{} is to assert that one
action occurs at least some time units after another, we assume that it is
possible to encode the same pattern using clock-style constraints.

\subsection{Messages~(\ref{sec:metamodel-messages})}

\paragraph{\mmessageset}
\begin{featset}
\item[PSC] \insp{constraint set}
\end{featset}

\mrefmessageset s are directly inspired by the PSC approach to referencing constraint sets.

\paragraph{\mmessagespec}
\begin{featset}
\item[UML] ??
\item[PSC] \insp{arrowMSG, intraMSG}
\end{featset}

We do not cover PSC required or fail messages.
Notion of direction (inbound/outbound) rather than sender/receiver labels.

\subsection{Actors~(\ref{sec:metamodel-actors})}

\paragraph{\mactor}
\begin{featset}
\item[UML] lifeline
\item[PSC] \insp{component instance}
\item[RC] \insp{module}
\end{featset}

Always fixed at two \ghtodo{32}{for now} in \langname: a \mtarget{} and a \mworld{}.
Neither are named, though the \mtarget{} will refer to a
named component such as a \mrcmodule.

Like PSC, but unlike UML, there are no executions.

\subsection{Matrix}

\newcommand\rot{\rotatebox{90}}
\newcommand\matding[1]{{\small#1}}
\newcommand\OK{\matding{\checkmark}}
\newcommand\ISH{\matding{(\OK)}}
\newcommand\NO{\matding{\(\bullet\)}}
\newcommand\SOON{\matding{\(\circ\)}}
\newcommand\NA{\matding{n/a}}

\begin{table}[htb!]
  \label{tab:seq-comparison-features}
  \centering

  \begin{tabular}{rl|ccccc|cc|ccc|c}
  \toprule

  & \rot{\thead{\langname}}
  & \rot{\thead{\featname{UML}}}
  & \rot{\thead{\featname{MARTE}}}
  & \rot{\thead{\featname{PSC}}}
  & \rot{\thead{\featname{TPSC}}}
  & \rot{\thead{\featname{PTPSC}}}
  & \rot{\thead{\featname{PSP}}}
  & \rot{\thead{\featname{AGLPT}}}
  & \rot{\thead{\featname{CSP}}}
  & \rot{\thead{\featname{PRISM}}}
  & \rot{\thead{\featname{IUTP}}}
  & \rot{\thead{\featname{RC}}}
  \\
  \midrule
  \multicolumn{12}{l}{\tsubhead{Messages}}
  \\
  Regular & \OK & \OK & \OK & \OK & \OK & \OK & ? & ? & \OK & ? & ? & \OK
  \\
  Expected & \NO & \NO & \NO & \OK & \OK & \OK & ? & ? & ? & ? & ? & \NA
  \\
  Fail & \NO & \NO & \NO & \OK & \OK & \OK & ? & ? & ? & ? & ? & \NA
  \\
  \midrule
  \multicolumn{12}{l}{\tsubhead{Waits}}
  \\
  Explicit (\mwaitaction) & \OK & \ISH & \ISH & \NO & \ISH? & \ISH? & ? & ? & \OK & ? & ? & \OK
  \\
  Ranged & \SOON & \ISH & \ISH & \NO & \ISH? & \ISH? & ? & ? & \OK & ? & ? & \OK
  \\
  \midrule
  \multicolumn{12}{l}{\tsubhead{Duration constraints}}
  \\
  Lower-bound & \SOON? & \OK & \OK & \NO & \OK? & \OK? & ? & ? & \ISH & ? & ? & ?
  \\
  Upper-bound/deadline & \OK & \OK & \OK & \NO & \OK & \OK & ? & ? & \OK & ? & ? & \OK
  \\
  \midrule
  \multicolumn{12}{l}{\tsubhead{Conditionally executed blocks}}
  \\
  Optional & \SOON? & \OK & \OK & \ISH & \ISH & \ISH & ? & ? & \OK & ? & ? & ?
  \\
  Probabilistic optional & \SOON & \NO & \NO & \NO & \NO & \OK & ? & ? & \NO & ? & ? & ?
  \\
  Alternative (if) & \SOON? & \OK & \OK & \OK & \OK & \OK & ? & ? & \OK & ? & ? & \OK
  \\
  \midrule
  \multicolumn{12}{l}{\tsubhead{Other}}
  \\
  Gaps (or similar) & \OK & \NO & \NO & \OK & \OK & \OK & ? & \ISH? & \ISH & ? & ? & \NA
  \\
  Loops (\mloopstep) & \OK & \OK & \OK & \OK & \OK & \OK & ? & ? & \OK & ? & ? & \NO
  \\
  Breaks & \SOON & \OK & \OK & \OK & \OK & \OK & ? & ? & \ISH & ? & ? & \NA
  \\
  \bottomrule
  \end{tabular}
  \caption{Matrix of features available in \langname{} sequences compared to
  other notations.  (Key: \OK{}={}supported; \ISH{}={}indirectly encodable; \SOON{}={}planned; \NO{}={}unsupported)}
\end{table}

\Cref{tab:seq-comparison-features} summarises both the features listed earlier
(which \langname{} has, and may correspond to similar features in other
notations) and features in other notations for which \langname{} does not,
or not yet, have a counterpart.

%%% Local Variables:
%%% mode: latex
%%% TeX-master: "../../robocert"
%%% End:
