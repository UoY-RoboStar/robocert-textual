% !TEX root=../../robocert.tex
\newcommand{\insp}[1]{\ul{#1}}

This section compares \langname{} sequence features to any
corresponding features in the other notations.
For each feature, we highlight the main source of inspiration (if any)
using \insp{underlines}.

%\subsection{Sequences~(\ref{sec:metamodel-sequences})}

\subsection{Steps~(\ref{sec:metamodel-steps})}

\paragraph{\mdeadlinestep}
\begin{featset}
\item[UML] \insp{duration constraint}, time observations and constraints
\item[TPSC] clock constraint
\item[AGLPT] upper time bounds
\end{featset}

UML2 allows time and duration constraints to relate to named observations taken
previously in the sequence.  By combining a time observation at one end of a
subsequence with a time constraint at the other, we can capture a form of clock
constraint.  We do not yet capture observations.

While \robochart{} has clocks, we do not yet expose them in
\langname, as we consider them to be an implementation concern.
      
\paragraph{\mloopstep}
\begin{featset}
\item[UML] \insp{loop combined fragment}
\item[PSC] loop operator
\end{featset}

\mloopbound s are inspired by those permitted by UML.

\paragraph{Gaps}
\begin{featset}
\item[PSC] \insp{constraints}, strict operator
\item[AGLPT] until
\end{featset}

Gaps resemble past-unwanted-message constraints, but
allow restricting the set of \emph{allowed} messages;
this subsumes the strict operator.  Graphical syntax is a slight
modification of PSC syntax.  We do not cover
future-unwanted-message or chain constraints.  Currently, all
ordering is assumed strict unless modified by a gap; this
deviates from PSC \emph{and} some readings of UML.
    
\subsection{Actions~(\ref{sec:metamodel-actions})}

\paragraph{\marrowaction}
\begin{featset}
\item[UML] \insp{message occurrence specification}
\item[PSC] regular message
\end{featset}

Arrows are named for the PSC \emph{arrowMSG} concept but are closer
to UML.
      
\paragraph{\mwaitaction}
\begin{featset}
\item[UML] combination of time observations and constraints(?)
\item[TPSC] clock constraints(?)
\end{featset}

The exact form of this operator comes mostly from \robochart{} and, to
a lesser extent, \tockcsp.
Because wait actions in \langname{} assert that one
action occurs at least some time units after another, we assume that it is
possible to encode the same pattern using clock-style constraints.

\subsection{Messages~(\ref{sec:metamodel-messages})}

\paragraph{\mmessageset}
\begin{featset}
\item[PSC] \insp{constraint set}
\end{featset}

\mrefmessageset s are directly inspired by the PSC approach to referencing constraint sets.

\paragraph{\mmessagespec}
\begin{featset}
\item[UML] ??
\item[PSC] \insp{arrowMSG, intraMSG}
\end{featset}

We do not cover PSC required or fail messages.
Notion of direction (inbound/outbound) rather than sender/receiver labels.

\subsection{Actors~(\ref{sec:metamodel-actors})}

\paragraph{\mactor}
\begin{featset}
\item[UML] lifeline
\item[PSC] \insp{component instance}
\end{featset}

Unlike UML and PSC, we fix the number of actors at two
\ghtodo{32}{for now} in \langname: a \mtarget{} and a \mworld{}.
Neither are named, though the \mtarget{} will refer to a
named component such as a \mrcmodule.

Like PSC, but unlike UML, there are no executions.

\subsection{Assertions~(\ref{sec:seq-metamodel-assertions})}

Being part of a language for automated property checking,
\langname{} sequence assertions more closely resemble CSP-M/FDR refinement
assertions than the usual ways of reasoning about UML or PSC sequence diagrams.

\subsection{Matrix}

\newcommand\rot{\rotatebox{90}}
\newcommand\matding[1]{{\small#1}}
\newcommand\OK{\matding{\checkmark}}
\newcommand\ISH{\matding{E}}
\newcommand\NO{\matding{\(\bullet\)}}
\newcommand\SOON{\matding{\(\circ\)}}
\newcommand\NA{\matding{n/a}}
\newcommand\INTIMED{\matding{T}}
\newcommand\INPROB{\matding{P}}

\begin{table}[htb!]
  \label{tab:seq-comparison-features}
  \centering

  \begin{tabular}{rlcccccc}
  \toprule

  & \rot{\thead{\langname}}
  & \rot{\thead{\featname{UML}}}
  & \rot{\thead{\featname{MARTE}}}
  & \rot{\thead{\featname{STAIRS}}}
  & \rot{\thead{\featname{PSC}}}
  & \rot{\thead{\featname{PSP}}}
  & \rot{\thead{\featname{AGLPT}}}
  \\
  \midrule
  \multicolumn{7}{l}{\tsubhead{Messages}}
  \\
  Asynchronous & \OK & \OK & \OK & \OK & \NO & ? & ?
  \\
  Synchronous & \SOON? & \OK & \OK & \OK & \OK & ? & ?
  \\
  Expected/fail & \NO & \NO & \NO & \NO & \OK & ? & ?
  \\
  \midrule
  \multicolumn{7}{l}{\tsubhead{Waits}}
  \\
  Explicit (\mwaitaction) & \OK & \ISH & \ISH & \INTIMED & \INTIMED? & ? & ?
  \\
  Ranged & \SOON & \ISH & \ISH & \INTIMED & \INTIMED? & ? & ?
  \\
  \midrule
  \multicolumn{7}{l}{\tsubhead{Duration constraints}}
  \\
  Lower-bound & \SOON? & \OK & \OK & \INTIMED & \INTIMED? & ? & ?
  \\
  Upper-bound/deadline & \OK & \OK & \OK & \INTIMED & \INTIMED & ? & ?
  \\
  \midrule
  \multicolumn{7}{l}{\tsubhead{Conditionally executed blocks}}
  \\
  Optional (\texttt{opt}) & \SOON? & \OK & \OK & \OK & \ISH & ? & ?
  \\
  Probabilistic optional & \SOON & \NO & \NO & \NO & \INPROB & ? & ?
  \\
  Potential alternative (\texttt{alt}, \(\sqcap\)) & \SOON & \OK & \OK & \OK & \OK & ? & ?
  \\
  Mandatory alternative (\texttt{xalt}, \(\extchoice\)) & \SOON? & \NO & \NO & \OK & \OK & ? & ?
  \\
  Probabilistic alternative (\texttt{palt}) & \SOON? & \NO & \NO & \INPROB & ? & ? & ?
  \\
  \midrule
  \multicolumn{7}{l}{\tsubhead{Other}}
  \\
  Gaps (or similar) & \OK & \NO & \NO & \NO & \OK & ? & \ISH?
  \\
  Loops (\mloopstep) & \OK & \OK & \OK & \OK & \OK & ? & ?
  \\
  Breaks & \SOON & \OK & \OK & \OK & \OK & ? & ?
  \\
  \bottomrule
  \end{tabular}
  \caption{Matrix of features available in \langname{} sequences compared to
  other notations.\\
  \small{(Key: \OK{}={}supported;
  \ISH{}={}indirectly encodable; 
  \INTIMED{}={}timed variant only;
  \INPROB{}={}probabilistic variant only;
  \SOON{}={}planned; \NO{}={}unsupported)}}
\end{table}

\Cref{tab:seq-comparison-features} summarises both the features listed earlier
(which \langname{} has, and may correspond to similar features in other
notations) and features in other notations for which \langname{} does not,
or not yet, have a counterpart.

%%% Local Variables:
%%% mode: latex
%%% TeX-master: "../../robocert"
%%% End:
