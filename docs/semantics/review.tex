%!TEX root=../robocert.tex

This chapter reviews known formal semantic treatments for any notations
mentioned earlier as similar to those in \langname, and compares them to
our approaches.

\section{Sequences}\label{sec:semantics-review-seq}

\subsection{UML2 and derived profiles}

\paragraph{CML semantics of SysML}

Our semantics (\cref{sec:semantics-tockcsp-seq}) takes heavy inspiration from
that of Lima et al.~\cite{lima-semantics} on the SysML profile of UML, with
some differences to account for a retargeting from CML to \tockcsp.

For two-\mactor{} sequences \ghtodo{32}{all of them at the moment}, we can (and do)
simplify the semantics significantly.  Other differences include:

\begin{itemize}
\item a focus on \tockcsp{} rather than CML, and consequently a
  treatment of timing properties based on the former;
\item
  the treatment of other concepts in \langname{} but not UML sequence
  diagrams (see \cref{sec:seq-comparison-features});
\item simplifications arising from the assumption that all sequences
  are between two actors whose actions are always either communications with
  each other or shared control-flow decisions.
\end{itemize}

The main novelty of our work here is that, by targeting \tockcsp, we can capture
timing properties and refinement relations.

\paragraph{MARTE}

We are not aware of a particular formal semantics for the MARTE profile.
\todo{Check}
There is work to generate HDL from MARTE sequence diagrams~\todo{cite}

\paragraph{Trace semantics of STAIRS}

STAIRS~\cite{Haugen03-STAIRS} gives a trace semantics for UML2 in terms
of pairs of positive and negative trace sets (the latter capturing the UML2
\texttt{neg} operator).  These traces contain
transmission and consumption events across signals.  STAIRS defines
refinement operators that are sensitive to the possibility that positive
traces can be recategorised as negative traces after refinement.  Our
semantics indirectly provides a similar semantics for \langname{}
sequences through \tockcsp{} (and its traces and \emph{tick-tock} models~\cite{Baxter21-TickTock}).

Timed
STAIRS~\cite{Haugen05-TimedSTAIRS} traces also contain separate reception
events (capturing the distinction between the time an event reaches
its destination and the time it is processed), and \(\mathbb R\)-valued
timestamps capturing timing.  We instead use the discrete \tockcsp{} notion
of time, and do not directly cater for the reception--consumption distinction
\todo{would CSP hiding and renaming account for this?}.

\subsection{PSC, TPSC, and PTPSC}

Property Sequence Chart (and its derivatives) is a layer atop
linear temporal logic, which means that its operational semantics is
in the form of B\"uchi automata (for TPSC, timed B\"uchi
automata; it is unclear what needs to be added to these automata for PTPSC \todo{?}).
There is 

For PSC, a \textsc{Charmy} plugin automates the operational semantics.  A denotational semantics in terms of
invalid traces also exists.