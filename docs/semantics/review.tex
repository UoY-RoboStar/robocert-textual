%!TEX root=../robocert.tex

This chapter reviews known formal semantic treatments for any notations
mentioned earlier as similar to those in \langname, and compares them to
our approaches.

\section{Sequences}\label{sec:semantics-review-seq}

This section reviews formal semantics development efforts for sequence diagram
notations.

\subsection{UML2 and derived profiles}

The standard semantics for UML2 sequence diagrams (see
\cite[\S 17.2.3.1]{uml251}) is a pair of \emph{positive} and \emph{negative}
traces, with an implicit notion of inconclusive traces belonging to
the universe but not to either set.  Deriving these sets involves the
fulfilment of partial orders over messages.

As UML2 expresses this
semantics in ambiguous natural language, and as the trace-pair
semantics is not always compatible with the diverse use cases of sequence
diagrams, there are several projects to provide an alternative formal semantics.

\paragraph{Micskei--Waeselynck survey}

In 2011, Micskei and Waeselynck~\cite{Micskei11-UMLSeqSemaSurvey} surveyed
the formal semantics for UML sequence diagrams available at the time.  These
efforts involve developments of the double-trace-set approach as well as other
denotational and operational semantics.

This study is useful here for two reasons.  First, it captures the then-state of
the art in UML sequence diagram formalisation.  Later on in this section,
we compare against some of the identified treatments (for instance,
\emph{STAIRS}).

Second, it provides a detailed
set of points (with thought experiments) where the formalisms diverge both from
each other and from UML.
This comparison shows that different use cases, interpretations, and backgrounds
have given rise to many different semantics for UML sequence diagrams, some
of which differ significantly from the standard.

The survey gives the comparison points further context by showing how a
UML2-derived sequence diagram language, \emph{TERMOS}, resolves each.
We can follow a similar process for \langname{} sequences, as follows:

\begin{description}
\item[Representing events]
  \langname{} represents events as (action, sender, receiver, message name)
  tuples at a metamodel level (see \cref{sec:metamodel-messages}),
  though for 2-actor sequences the sender and
  receiver are implicit in the direction, and in the \tockcsp{} semantics a
  notion of typed channels also renders some aspects of the encoding implicit.
  There is no way to distinguish, given two identical sends occurring at the
  same instant in time, which was responsible for which corresponding receive;
  we do not think that this distinction is important for our use case
  \todo{is it?}
\item[Categorising traces]
  \langname{} is a verification language and (contravening UML2)
  uses a two-category approach: valid/other or invalid/other, depending on the
  \metafeature{isNegated} flag on the \massertion.
  As the main \langname{} semantics is \tockcsp, we use a
  refinement-style categorisation where a sequence denotes an
  expected set of traces (or stable failures, and so on) that must be a superset
  of those calculated from the target.
\item[Complete or partial traces]
  \langname{} sequences (contravening UML2) are \emph{partial} with, by default,
  any suffix permitted.  Prefixes may be permitted by using gaps at the start of
  the sequence.  This difference reflects the underlying trace model of \tockcsp.
  \todo{Ana mentioned a good reason for why we might want this semantics that
  isn't just `CSP does it', but I can't remember precisely what it was.
  Compositionality?}
\item[Combining fragments]
  \langname{} sequences (contravening UML2) currently synchronise on entering
  their equivalent of combined fragments.  \todo{This doesn't actually matter
  yet for 2-actor sequences.}
\item[Processing the diagram]
  \langname{} sequences tend to have a compositional semantics whereby metamodel
  concepts translate bottom-up into fragments of
  the object language, with combined fragments translating into the relevant
  operators,
  and additional superstructure such as memory processes and parameterisations
  being added as needed.
\item[Underlying formalism: Approach]
  The \langname{} semantics maps each sequence to a representation in another
  language, to which we delegate the choice of approach.
  In at least the example of \tockcsp, the end result is a calculation of the
  set of all possible traces (or similar denotation), which corresponds at
  least partially to UML2.
\item[Underlying formalism: Concurrency]
  \langname{} sequences have an interleaving semantics, in accordance with
  both UML2 and \tockcsp.
\item[Handling choices]
  Not set yet
  \todo{I suspect this will create a global time point for the choice.}
\item[Interpretation of a false guard]
  Not set yet.
  \todo{I suspect this will act as in CSP eg pulling the alternative down to
  \(\Stop\).}
\item[Who evaluates a guard]
  Not set yet.
  \todo{I suspect this would be the memory process, which would suggest a
  global evaluation.}
\item[Formal and actual Gates]
  These do not exist in \langname{} sequences.
\item[Interpretation of \texttt{neg(S)}]
  These do not exist in \langname{} sequences.
\item[Ignore/consider]
  These do not exist in \langname{} sequences.
\item[Conformance-related operators in complex diagrams]
  \langname{} sequences have no conformance-related operators at the diagram
  level; sequences always represent a set of possible traces (or similar
  denotation) to be compared against those observable from a model.
\item[Traces being both valid and invalid]
  Theoretically, all traces are either allowed or disallowed by the diagram,
  with the lack of features for characterising traces within the diagram itself
  serving as a restriction to enforce this.  It may not
  be decidable to verify that all traces exhibitable by a model are allowed by
  the diagram (or that one trace disallowed by the diagram is never observable
  in the model).
\end{description}

\paragraph{CML semantics of SysML}

Our semantics (\cref{sec:semantics-tockcsp-seq}) takes heavy inspiration from
that of Lima et al.~\cite{lima-semantics} on the SysML profile of UML, with
some differences to account for a retargeting from CML to \tockcsp.

For two-\mactor{} sequences \ghtodo{32}{all of them at the moment}, we can (and do)
simplify the semantics significantly.  Other differences include:

\begin{itemize}
\item a focus on \tockcsp{} rather than CML, and consequently a
  treatment of timing properties based on the former;
\item
  the treatment of other concepts in \langname{} but not UML sequence
  diagrams (see \cref{sec:seq-comparison-features});
\item simplifications arising from the assumption that all sequences
  are between two actors whose actions are always either communications with
  each other or shared control-flow decisions.
\end{itemize}

The main novelty of our work here is that, by targeting \tockcsp, we can capture
timing properties and refinement relations.

\paragraph{MARTE}

We are not aware of a particular formal semantics for the MARTE profile.
\todo{Check}
There is work to generate HDL from MARTE sequence diagrams~\todo{cite}

\paragraph{Trace semantics of STAIRS}

STAIRS~\cite{Haugen03-STAIRS} gives a trace semantics for UML2 in terms
of pairs of positive and negative trace sets (the latter capturing the UML2
\texttt{neg} operator).  These traces contain
transmission and consumption events across signals.  STAIRS defines
refinement operators that are sensitive to the possibility that positive
traces can be recategorised as negative traces after refinement.

Timed
STAIRS~\cite{Haugen05-TimedSTAIRS} traces also contain separate reception
events (capturing the distinction between the time an event reaches
its destination and the time it is processed), and \(\mathbb R\)-valued
timestamps capturing timing.  We instead use the discrete \tockcsp{} notion
of time, and do not directly cater for the reception--consumption distinction
\todo{would CSP hiding and renaming account for this?}.

Our work indirectly provides a similar semantics for \langname{}
sequences through \tockcsp{} (and its traces and \emph{tick-tock} models~\cite{Baxter21-TickTock}).
Our approach achieves automation through existing tools such as FDR,
and is compatible with the existing \robostar{} ecosystem as well as that of
\tockcsp.

\subsection{PSC, TPSC, and PTPSC}

Property Sequence Chart (and its derivatives) is a layer atop
linear temporal logic, which means that its operational semantics is
in the form of B\"uchi automata (for TPSC, timed B\"uchi
automata; it is unclear what needs to be added to these automata for PTPSC \todo{?}).
There is 

For PSC, a \textsc{Charmy} plugin automates the operational semantics.  A denotational semantics in terms of
invalid traces also exists.