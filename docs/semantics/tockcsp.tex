%!TEX root=../robocert.tex

This section introduces a \tockcsp{} semantics for \langname.
\todo{Need to make sure it's actually \tockcsp, not CSP.}
This semantics is inspired by that of Lima et al. on the CML semantics of
UML sequence diagrams.

The behaviour of the \langname{} generator is expected to conform to
this semantics, except that:

\begin{itemize}
\item
	The generator targets \cspm{} instead of CSP, and so \emph{may}
	use semantically equivalent \cspm{} constructs where the CSP equivalents
	are missing or not idiomatic;
\item
	The generator \emph{must} \todo{not yet but eventually} wrap processes in
	timed sections to achieve the appropriate \tockcsp{} behaviours of
	CSP operators;
\item
	The generator \emph{may} perform semantics-preserving optimisations,
	such as substituting \(P\) for \(\tlang{\Stop \interrupt \olang{P}}\).
\end{itemize}

\todo{Some of the definitions are very loosely typed (as they are expanding from
a metamodel to textual snippets of something halfway between \tockcsp{} and
\cspm), and it'd be nice to fix that and/or give explicit result types.}

\subsection{Assertions}

The definitions here correspond to \cref{sec:metamodel-assertions}.

\begin{defn}[\massertion]

\newcommand{\refop}[3]{\mathbin{\odot_{#1}^{(#2, #3)}}}

The only type of assertion so far is \msequenceassertion.  The semantics of a
\msequenceassertion{} relates the \msequence{} of the assertion to the
\mtarget{} of the assertion.
%
\begin{align*}
	\asstsema{\asasst}
\quad\defeq\quad&
	\seqnameOf{\field{\asasst}{sequence}}
	\;
	\refop{\field{\asasst}{model}}{\field{\asasst}{isNegated}}{\field{\asasst}{type}}
	\;
	\targetsema{\field{\field{\asasst}{sequence}}{target}}{\field{\asasst}{instantiation}}
\end{align*}

The exact refinement operator depends on the assertion type and negation:
%
\begin{align*}
	\refop{\amodel}{\true}{\mathsf{holds}}
\quad\defeq\quad&
	\tlang{\sqsubseteq_{\olang{\amodel}}}
&
	\refop{\amodel}{\false}{\mathsf{holds}}
\quad\defeq\quad&
	\tlang{\not\sqsubseteq_{\olang{\amodel}}}
\\
	\refop{\amodel}{\true}{\mathsf{isObserved}}
\quad\defeq\quad&
	\tlang{\sqsupseteq_{\olang{\amodel}}}
&
	\refop{\amodel}{\false}{\mathsf{isObserved}}
\quad\defeq\quad&
	\tlang{\not\sqsupseteq_{\olang{\amodel}}}
\\
\end{align*}
\end{defn}


\subsection{Sequences}\label{ssec:semantics-tockcsp-sequences}

The definitions here correspond to \cref{sec:metamodel-sequences}.

\begin{defn}[\msequence]

The semantics of a \msequence{} is that of its subsequence
\todo{eventually, in parallel with its memory}.
%
\begin{align*}
	\seqsema{\aseq}
\quad\defeq\quad&	
	\sseqsema{\field{\aseq}{body}}
\end{align*}

\end{defn}

\begin{defn}[\msubsequence]

The semantics of a \msubsequence{} is a sequential composition of that of its steps.
%
\begin{align*}
	\sseqsema{\asseq}
	\quad\defeq\quad&	
	\funcname{steps}(\field{\asseq}{body})
\\
	\funcname{steps}(\langle \astep_1, \dotsc, \astep_n \rangle)
	\quad\defeq\quad&	
	\tlang{
	\olang{\stepsema{\astep_1}}
	\circseq
	\olang{\dotso}
	\circseq
	\olang{\stepsema{\astep_n}}
	}
\end{align*}

\end{defn}

\begin{defn}[\msequencestep]

The semantics of a \msequencestep{} is an unbounded loop over the events of its
\msequencegap, interrupted by the \msequenceaction.\footnote{Note that the semantics of the gap depends
on the action.  This is because, as seen in \(\gapsema{\agap}{\anaction}\),
the events on which the action can interrupt the
gap must be removed from the events of the gap to ensure the interrupt has the
expected semantics.}
%
\begin{align*}
	\stepsema{\astep}
\quad\defeq\quad&	
	\tlang{
		\runproc{\olang{\gapsema{\field{\astep}{gap}}{\field{\astep}{action}}}}
		\interrupt \olang{\actsema{\field{\astep}{action}}}
	}
\end{align*}
\end{defn}

\begin{defn}[\msequencegap]
	The semantics of a \msequencegap{} is the CSP event set corresponding to
	the difference between the \emph{allowed} set,
	and the result of extending the \emph{forbidden} set with any events
	the action following the gap can initially communicate.
%
\begin{align*}
	\gapsema{
		\agap
	}{\anaction}
\quad\defeq\quad&
\tlang{
	\olang{\msgsetsema{\field{\agap}{allowed}}}
	\setminus
	\left(
		\olang{\msgsetsema{\field{\agap}{forbidden}}}
		\cup
		\olang{\eventsOf{\anaction}}
	\right)
}
\end{align*}
\end{defn}

\begin{defn}[CSP event sets]
For now, the event set of an action is the prefix of an arrow action, or
the empty set for any other actions.  This will change when the prefix becomes
inexpressible as an event set.
%
\begin{align*}
	\eventsOf{\anarrow}
\quad\defeq\quad&
	\tlang{\Set{\olang{\emspecsema{\anarrow}}}}
	\tag{arrow}
\\
	\eventsOf{\anaction}
\quad\defeq\quad&
	\tlang{\emptyset}
	\tag{anything else}
\end{align*}
\end{defn}

\subsection{Actions}\label{ssec:semantics-tockcsp-actions}

The definitions here correspond to \cref{sec:metamodel-actions}.

\begin{defn}[\msequenceaction]

The semantics of an arrow action is the semantics of its arrow as a CSP prefix,
prefixing termination.  The semantics of a loop delegates to a separate rule
over its label and subsequence.
%
\begin{align*}
	\actsema{\anarrow}
\quad\defeq\quad&
	\tlang{
	\left(
	\olang{\pmspecsema{\anarrow}}
	\then
	\Skip
	\right)
	}
	\tag{arrow action}
\\
	\actsema{\aloop}
\quad\defeq\quad&
	\loopsema{\aloop}
\tag{loop action}
\\
	\actsema{\bot}
\quad\defeq\quad&
	\tlang{\Skip}
\tag{final action}
\end{align*}

\end{defn}

\begin{defn}[\mloopaction]

For now, only infinite loops exist, and their semantics is that of the loop
subsequence placed in a recursive process.
%
\begin{align*}
	\loopsema{\aloop}
\quad\defeq\quad&
\tlang{
	\circmu \mathsf{LOOP}_{\olang{\field{\aloop}{name}}} \bullet
	\left(
		\olang{\sseqsema{\field{\aloop}{body}}}
		\circseq \mathsf{LOOP}_{\olang{\field{\aloop}{name}}}
	\right)
}
\tag{infinite loop}
\end{align*}

\end{defn}

\subsection{Messages}\label{ssec:semantics-tockcsp-messages}

The definitions here correspond to \cref{sec:metamodel-messages}.

\begin{defn}[\mmessageset]

TODO
%
\begin{align*}
	\msgsetsema{\aumsgset}
\quad\defeq\quad&
	\tlang{\events}
\tag{universe}
\\
\intertext{\todo{The meta/object language distinction is messy here
and I'm not sure how to handle it.}}
	\msgsetsema{\anemsgset}
\quad\defeq\quad&
\tlang{
	\bigcup
\olang{
	\Set{
		\emspecsema{\amspec} | \amspec \in \field{\anemsgset}{messages}
	}
}
}
\tag{extensional}
\end{align*}
\end{defn}

There are two distinct semantic rules for \mmessagespec s.  One is used whenever
we expand the \mmessagespec{} into an event set, and handles `rest' arguments
by implicitly quantifying over the missing parameters.  The other is used when
expanding to a prefix, and introduces explicit discards for the missing
parameters.  \todo{The two forms will diverge further when we introduce
parameter binding.}

\begin{defn}[\mmessagespec{} as an event set]

We expand \mmessagespec s into event-set form in two stages: expanding
any non-`rest' arguments with \funcname{withArgs}, then expanding the channel
reference using \funcname{msgChan}.
%
\begin{align*}
	\emspecsema{\amspec}
\quad\defeq\quad&
\funcname{withArgs}\left(\field{\amspec}{arguments}, \amspec\right)
\\
	\funcname{withArgs}\left(\amspec, \anarglist\cat\langle\anexprarg\rangle\right)
\quad\defeq\quad&
\tlang{
	\olang{\funcname{withArgs}\left(\amspec, \anarglist\right)}
	!\olang{\exprsema{\anexprarg}{\amspec}}
}
\tag{expression argument}
\\
	\funcname{withArgs}\left(\amspec, \anarglist\cat\langle\arestarg\rangle\right)
\quad\defeq\quad&
	\funcname{withArgs}\left(\amspec, \anarglist\right)
\tag{rest argument}
\\
	\funcname{withArgs}\left(\amspec, \langle\rangle\right)
\quad\defeq\quad&
	\funcname{msgChan}\left(\amspec\right)
\tag{end of arguments}
\end{align*}
The expansion of the message channel
depends on the message topic
and, for certain topics, the direction (inbound from world to target, or
outbound from target to world).
\newcommand{\nsOf}[1]{\mathsf{ns}(#1)}
\newcommand{\targetOf}[2]{\mathsf{target}(#1,#2)}
\newcommand{\topicOf}[3]{\mathsf{topic}(#1,#2,#3)}
%
\begin{align*}
	\funcname{msgChan}(\amspec)
\quad\defeq\quad&
\tlang{
	\olang{\nsOf{\targetOf{\field{\amspec}{from}}{\field{\amspec}{to}}}}
	{}\cspnsop{}
	\olang{\topicOf{\field{\amspec}{topic}}{\field{\amspec}{from}}{\field{\amspec}{to}}}
}
\\
	\targetOf{\aworld}{\atarget}
\quad\defeq\quad&
	\targetOf{\atarget}{\aworld}
	\quad\defeq\quad
	\atarget
\\
	\topicOf{\anop}{\atarget}{\aworld}
\quad\defeq\quad&
\tlang{
	\olang{\field{\anop}{name}}\text{Call}
}
\tag{operation; always outbound}
\\
	\topicOf{\anevent}{\aworld}{\atarget}
\quad\defeq\quad&
	\tlang{\olang{\field{\anevent}{name}}\text{.in}}
\tag{inbound event}
\\
	\topicOf{\anevent}{\atarget}{\aworld}
\quad\defeq\quad&
	\tlang{\olang{\field{\anevent}{name}}\text{.out}}
\tag{outbound event}
\end{align*}
\end{defn}

\newcommand{\paramsOf}[1]{\funcname{params}(#1)}
\newcommand{\beforeRest}[1]{\funcname{beforeRest}(#1)}

\begin{defn}[\mmessagespec{} as a prefix]

To expand a \mmessagespec{} into a prefix, we take the event set expansion and
\todo{for now} \emph{pad} it with wildcard inputs for each topic parameter
left unmatched at the point of any \mrestargument.
%
\begin{align*}
	\pmspecsema{\amspec}
\quad\defeq\quad&
	\funcname{pad}\left(\emspecsema{\amspec}, \funcname{padding}(\amspec)\right)
\\
	\funcname{pad}(x, 0)
\quad\defeq\quad&
	x
\tag{base case}
\\
	\funcname{pad}(x, n)
\quad\defeq\quad&
\tlang{
	\olang{\funcname{pad}(x, n-1)}?\_
}
\tag{inductive case}
\\
	\funcname{padding}(\amspec)
\quad\defeq\quad&
	\#(\paramsOf{\field{\amspec}{topic}}) - \beforeRest{\field{\amspec}{arguments}}
\\
	\beforeRest{\langle\rangle}
\quad\defeq\quad&
	0
\tag{base case}
\\
	\beforeRest{\langle \arestarg \rangle \cat \anarglist}
\quad\defeq\quad&
	0
\tag{rest argument}
\\
	\beforeRest{\langle \anarg \rangle \cat \anarglist}
\quad\defeq\quad&
	1 + \beforeRest{\anarglist}
\tag{non-rest argument}
\end{align*}
\end{defn}

\subsection{Actors}\label{ssec:semantics-tockcsp-actors}

The definitions here correspond to \cref{sec:metamodel-actors}.


\begin{defn}[\mtarget]

The semantics of a target is the parametric process generated for that
target by the relevant external semantics, with constant parameters instantiated
as follows:

\begin{itemize}
\item
	if the constant is bound in the assertion's \mtargetinstantiation{}
	(passed to the semantics as \(\aninst\)), use the bound expression; else
\item
	if the constant is bound in the target's \mtargetinstantiation, use
	the bound expression; else
\item
	use the \(\funcname{constName}\) of the constant, under the assumption
	that it will later bind to a fallback instantiation for the constant
	\todo{review this - it seems strange}.
\end{itemize}
%
\begin{align*}
	\targetsema{\atarget}{\aninst}
\quad\defeq\quad&
\tlang{
	\olang{\funcname{targetProcess}(\atarget)}
	\left(
		\olang{\exprsema{-}{\atarget}\ \circ\ \funcname{instantiate}(\aninst)\ \circ\ \funcname{targetParams}(\atarget)}
	\right)
}
\\
	\funcname{instantiate}(\aninst, \atarget)
\quad\defeq\quad&
	\funcname{constName}
	\oplus
	\field{\field{\atarget}{instantiation}}{constants}
	\oplus
	\field{\aninst}{constants}
\end{align*}
\end{defn}


%%% Local Variables:
%%% mode: latex
%%% TeX-master: "../robocert"
%%% End:
