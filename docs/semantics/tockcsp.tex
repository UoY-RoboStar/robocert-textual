%!TEX root=../robocert.tex

This section introduces a semantics for \langname{} in terms of
\emph{tock}-CSP.
This semantics borrows heavily from that of Lima et al. on the CML semantics of
UML sequence diagrams.

The behaviour of the \langname{} generator is expected to conform to
this semantics, except that:

\begin{itemize}
\item
	The generator targets \cspm{} instead of CSP, and so \emph{may}
	use semantically equivalent \cspm{} constructs where the CSP equivalents
	are missing or not idiomatic;
\item
	The generator \emph{may} \todo{not yet but eventually} wrap processes in
	timed sections to achieve the appropriate \tockcsp{} behaviours of
	CSP operators;
\item
	The generator \emph{may} perform semantics-preserving optimisations,
	such as substituting \(P\) for \(\Stop \interrupt P\).
\end{itemize}

\subsection{Packages}\label{ssec:semantics-tockcsp-top}

The definitions here correspond to \cref{sec:metamodel-top}.

\todo{TODO}

\subsection{Sequences}\label{ssec:semantics-tockcsp-sequences}

The definitions here correspond to \cref{sec:metamodel-sequences}.

\begin{defn}[\msequence]

The semantics of a \msequence{} is, for now, that of its subsequence.

This rule does not capture the fact that the target and world definitions
introduce bindings referred to in message specs; for instance,
\lstinline[language=RoboCert]{sequence X for module Mod as M, world as W}
binds \lstinline[language=RoboCert]{M} to the module target,
\lstinline[language=RoboCert]{Mod} to its module, and
\lstinline[language=RoboCert]{W} to the world.
\todo{add target introduction}
%
\begin{align*}
	\seqsema{\aseq}
\quad\defeq\quad&	
\left(
	\seqnameOf{\aseq}{name} = \sseqsema{\field{\aseq}{body}}
\right)
\\
	\seqnameOf{\aseq}
\quad\defeq\quad&
	\field{\aseq}{name} \cspnsop \text{Sequence}
\end{align*}

\end{defn}

\begin{defn}[\msubsequence]

The semantics of a \msubsequence{} is a sequential composition of that of its steps.
%
\begin{align*}
	\sseqsema{\asseq}
	\quad\defeq\quad&	
	\mathsf{Steps}(\field{\asseq}{body})
\\
	\mathsf{Steps}(\langle\astep\rangle)
	\quad\defeq\quad&	
	\stepsema{\astep}
	\tag{base}
\\
	\mathsf{Steps}(\langle\astep\rangle\cat\mathbf{\astep})
	\quad\defeq\quad&	
	\stepsema{\astep} \circseq \mathsf{Steps}(\mathbf{\astep})
	\tag{inductive}
\end{align*}

\end{defn}

\begin{defn}[\msequencestep]

The semantics of a \msequencestep{} is an unbounded loop over the events of its
\msequencegap, interrupted by the \msequenceaction.\footnote{Note that the semantics of the gap depends
on the action.  This is because, as seen in \(\gapsema{\agap}{\anaction}\),
the events on which the action can interrupt the
gap must be removed from the events of the gap to ensure the interrupt has the
expected semantics.}
%
\begin{align*}
	\stepsema{\astep}
\quad\defeq\quad&	
	\runproc{\gapsema{\field{\astep}{gap}}{\field{\astep}{action}}}
	\interrupt \actsema{\field{\astep}{action}}
\end{align*}
\end{defn}

\begin{defn}[\msequencegap]
	The semantics of a \msequencegap{} is the CSP event set corresponding to
	the difference between the \emph{allowed} set,
	and the result of extending the \emph{forbidden} set with any events
	the action following the gap can initially communicate.
%
\begin{align*}
	\gapsema{
		\agap
	}{\anaction}
\quad\defeq\quad&
	\msgsetsema{\field{\agap}{allowed}}
	\setminus
	\left(\msgsetsema{\field{\agap}{forbidden}} \cup \eventsof{\anaction}\right)
\end{align*}
\end{defn}

\begin{defn}[CSP event sets]
For now, the event set of an action is the prefix of an arrow action, or
the empty set for any other actions.  This will change when the prefix becomes
inexpressible as an event set.
%
\begin{align*}
	\eventsof{\anarrow}
	\quad\defeq\quad&
	\Set{\mspecsema{\anarrow}}
	\tag{arrow}
\\
	\eventsof{\anaction}
	\quad\defeq\quad&
	\emptyset
	\tag{anything else}
\end{align*}
\end{defn}

\begin{defn}[\mgapmessageset]

Note that, per \cref{ssec:metamodel-sequences-gaps}, gap message sets in the
\emph{allowed} position can be either \mextensionalgapmessageset s or
\muniversegapmessageset s; we treat both using the same rule.
%
\begin{align*}
	\msgsetsema{\universe}
\quad\defeq\quad&
	\events
\tag{universe}
\\
\intertext{\todo{For now, the semantics of an \mextensionalgapmessageset{} depends on treating each
message
spec as if it were an \marrowmessagespec.  This will change when arrow and gap
message specs diverge.}}
	\msgsetsema{\amsgset}
\quad\defeq\quad&
	\bigcup \Set{\mspecsema{\amspec} | \amspec \in \amsgset }
\tag{extensional}
\end{align*}
\end{defn}

\subsection{Actions}\label{ssec:semantics-tockcsp-actions}

The definitions here correspond to \cref{sec:metamodel-actions}.

\begin{defn}[\msequenceaction]

The semantics of an arrow action is the semantics of its arrow as a CSP prefix,
prefixing termination.  The semantics of a loop delegates to a separate rule
over its label and subsequence.
%
\begin{align*}
	\actsema{\anarrow}
	\quad\defeq\quad&
	\left(
	\mspecsema{\anarrow}
	\then
	\Skip
	\right)
	\tag{arrow action}
\\
	\actsema{\aloop}
\quad\defeq\quad&
	\loopsema{\aloop}
\tag{loop action}
\\
\intertext{For now, a \mfinalaction{} does not terminate.}
	\actsema{\bot}
\quad\defeq\quad&
	\Stop
\tag{final action}
\end{align*}

\end{defn}

\begin{defn}[\mloopaction]

For now, only infinite loops exist, and their semantics is that of the loop
subsequence placed in a recursive process.
%
\begin{align*}
	\loopsema{\aloop}
\quad\defeq\quad&
	\circmu \mathsf{LOOP}_{\field{\aloop}{name}} \bullet
	\left(
		\sseqsema{\field{\aloop}{body}}
		\circseq \mathsf{LOOP}_{\field{\aloop}{name}}
	\right)
	\tag{infinite loop}
\end{align*}

\end{defn}

\subsection{Messages}\label{ssec:semantics-tockcsp-messages}

The definitions here correspond to \cref{sec:metamodel-messages}.

\begin{defn}[\mmessagespec]

The semantics of \mmessagespec s\footnote{For now, \mgapmessagespec s and
\marrowmessagespec s are equivalent in semantics; this may change later on.}
depends on the topic of their message specs,
and, for certain topics, the direction (inbound from world to target, or
outbound from target to world).  For now, neither operations nor events can have
arguments.

\newcommand{\nsOf}[1]{\mathsf{ns}(#1)}
\newcommand{\targetOf}[2]{\mathsf{target}(#1,#2)}
\newcommand{\topicOf}[3]{\mathsf{topic}(#1,#2,#3)}

Let \(\nsOf{\atarget}\) be the name of the module, controller, or state machine
referred to by \(\atarget\).
%
\begin{align*}
	\mspecsema{\anarrow}
\quad\defeq\quad&
	\nsOf{\targetOf{\field{\anarrow}{from}}{\field{\anarrow}{to}}}
	\cspnsop
	\topicOf{\field{\anarrow}{topic}}{\field{\anarrow}{from}}{\field{\anarrow}{to}}
\\
	\targetOf{\aworld}{\atarget}
\quad\defeq\quad&
	\targetOf{\atarget}{\aworld}
	\quad\defeq\quad
	\atarget
\\
	\topicOf{\anop}{\atarget}{\aworld}
\quad\defeq\quad&
	\field{\anop}{name}\ \mathsf{Call}
\\
	\topicOf{\anevent}{\aworld}{\atarget}
\quad\defeq\quad&
	\field{\anevent}{name}.\mathsf{in}
\\
	\topicOf{\anevent}{\atarget}{\aworld}
\quad\defeq\quad&
	\field{\anevent}{name}.\mathsf{out}
\end{align*}

\end{defn}

\subsection{Actors}

The definitions here correspond to \cref{sec:metamodel-actors}.

There are two rules for \mtarget{}s, corresponding to whether a
\emph{definition} or a \emph{reference} is required for the
\mtarget.\footnote{The CSP generator makes further distinction between
\emph{open} and \emph{closed} targets, but the latter is an optimising
implementation detail expressible in terms of the former.}
The target definition applies the \mtargetinstantiation{} of the \mtarget{}
to the parametric CSP process provided for it by the relevant semantics
(for instance, we delegate to the \robochart{} semantics for the underlying
\mrcmodule{} of a \mrcmoduletarget).

\begin{defn}[\mtarget{} definition]
	\todo{TODO}
\end{defn}

\begin{defn}[\mtarget{} reference]
	\todo{TODO}
\end{defn}
\begin{defn}[\mtarget{} closed reference]
	\todo{TODO}
\end{defn}

\begin{defn}[\mtarget{} closed definition]
	\todo{TODO}
\end{defn}

\subsection{Assertions}

The definitions here correspond to \cref{sec:metamodel-assertions}.

\begin{defn}[\massertion]

The only type of assertion so far is \msequenceassertion.

\newcommand{\refop}[3]{\mathbin{\odot_{#1}^{(#2, #3)}}}

\begin{align*}
	\asstsema{\asasst}
\quad\defeq\quad&
	\seqnameOf{\field{\asasst}{sequence}}
	\refop{\field{\asasst}{model}}{\field{\asasst}{isNegated}}{\field{\asasst}{type}}
\\
\intertext{The exact refinement operator depends on the assertion type and
negation:}
	\refop{\amodel}{\true}{\mathsf{holds}}
\quad\defeq\quad&
	\sqsubseteq_\amodel
\tag{holds}
\\
	\refop{\amodel}{\false}{\mathsf{holds}}
\quad\defeq\quad&
	\not\sqsubseteq_\amodel
\tag{does not hold}
\\
	\refop{\amodel}{\true}{\mathsf{isObserved}}
\quad\defeq\quad&
	\sqsupseteq_\amodel
\tag{is observed}
\\
	\refop{\amodel}{\false}{\mathsf{isObserved}}
\quad\defeq\quad&
	\not\sqsupseteq_\amodel
\tag{is not observed}
\\
\end{align*}
\end{defn}