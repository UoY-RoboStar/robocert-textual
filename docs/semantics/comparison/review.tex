%!TEX root=../../robocert.tex

This section reviews known formal semantic treatments for any notations
mentioned earlier as similar to those in \langname, and compares them to
our approaches.

\subsection{Sequences}\label{sec:semantics-comparison-review-seq}

This section reviews formal semantics development efforts for sequence diagram
notations.

\paragraph{Micskei--Waeselynck survey of UML2}

The standard semantics for UML2 sequence diagrams (see
\cite[\S 17.2.3.1]{uml251}) is a pair of \emph{positive} and \emph{negative}
traces, with an implicit notion of inconclusive traces belonging to
the universe but not to either set.  Deriving these sets involves the
fulfilment of partial orders over messages.

As UML2 expresses this
semantics in ambiguous natural language, and as the trace-pair
semantics is not always compatible with the diverse use cases of sequence
diagrams, there are several projects to provide an alternative formal semantics.

In 2011, Micskei and Waeselynck~\cite{Micskei11-UMLSeqSemaSurvey} surveyed
the formal semantics for UML sequence diagrams available at the time.  These
efforts involve developments of the double-trace-set approach as well as other
denotational and operational semantics.

This study is useful here for two reasons.  First, it captures the then-state of
the art in UML sequence diagram formalisation.  Later on in this section,
we compare against some of the identified treatments (for instance,
\emph{STAIRS}).

Second, it provides a detailed
set of points (with thought experiments) where the formalisms diverge both from
each other and from UML.
This comparison shows that different use cases, interpretations, and backgrounds
have given rise to many different semantics for UML sequence diagrams, some
of which differ significantly from the standard.

\paragraph{CML semantics of SysML}

Our semantics (\cref{sec:semantics-tockcsp-seq}) takes heavy inspiration from
that of Lima et al.~\cite{lima-semantics} on the SysML profile of UML, with
some differences to account for a retargeting from CML to \tockcsp.  This
semantics postdates the survey above.

For two-\mactor{} sequences \ghtodo{32}{all of them at the moment}, we can (and do)
simplify the semantics significantly.  Other differences include:

\begin{itemize}
\item a focus on \tockcsp{} rather than CML, and consequently a
  treatment of timing properties based on the former;
\item
  the treatment of other concepts in \langname{} but not UML sequence
  diagrams (see \cref{sec:seq-comparison-features});
\item simplifications arising from the assumption that all sequences
  are between two actors whose actions are always either communications with
  each other or shared control-flow decisions.
\end{itemize}

The main novelty of our work here is that, by targeting \tockcsp, we can capture
timing properties and refinement relations.

\paragraph{MARTE}

We are not aware of a particular formal semantics for the MARTE profile.
\todo{Check}
There is work to generate HDL from MARTE sequence diagrams~\todo{cite}

\paragraph{Trace semantics of STAIRS}

STAIRS~\cite{Haugen03-STAIRS} gives a trace semantics for UML2 in terms
of pairs of positive and negative trace sets (the latter capturing the UML2
\texttt{neg} operator).  These traces contain
transmission and consumption events across signals.  STAIRS defines
refinement operators that are sensitive to the possibility that positive
traces can be recategorised as negative traces after refinement.

Timed
STAIRS~\cite{Haugen05-TimedSTAIRS} traces also contain separate reception
events (capturing the distinction between the time an event reaches
its destination and the time it is processed), and \(\mathbb R\)-valued
timestamps capturing timing.  We instead use the discrete \tockcsp{} notion
of time, and do not directly cater for the reception--consumption distinction
\todo{would CSP hiding and renaming account for this?}.

Our work indirectly provides a similar semantics for \langname{}
sequences through \tockcsp{} (and its traces and \emph{tick-tock} models~\cite{Baxter21-TickTock}).
Our approach achieves automation through existing tools such as FDR,
and is compatible with the existing \robostar{} ecosystem as well as that of
\tockcsp.

\paragraph{PSC and derivatives}

Property Sequence Chart (and its derivatives) is a layer atop
linear temporal logic, which means that its operational semantics is
in the form of B\"uchi automata.\footnote{
Micskei and Waeselynck also note that several semantic treatments for UML2 also
target B\"uchi automata.}
For TPSC, the target is timed B\"uchi
automata; it is unclear what needs to be added to these automata for PTPSC \todo{?}.


For PSC, a \textsc{Charmy} plugin automates the operational semantics.  A denotational semantics in terms of
invalid traces also exists.