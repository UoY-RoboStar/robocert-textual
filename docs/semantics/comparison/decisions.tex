%!TEX root=../../robocert.tex

\todo{Non-sequence languages}

\subsection{Sequences}

The survey by Micskei and Waeselynck gives identifies many decision points that
distinguish derivations of semantics for UML and UML-style sequence diagram
languages, with a worked example in the form of the
UML2-derived sequence diagram language \emph{TERMOS}.
We can follow a similar process for \langname{} sequences, as follows:

\begin{description}
\item[Representing events]
  \langname{} represents events as (action, sender, receiver, message name)
  tuples at a metamodel level (see \cref{sec:metamodel-messages}),
  though for 2-actor sequences the sender and
  receiver are implicit in the direction, and in the \tockcsp{} semantics a
  notion of typed channels also renders some aspects of the encoding implicit.
  There is no way to distinguish, given two identical sends occurring at the
  same instant in time, which was responsible for which corresponding receive;
  though both Lima et al. and Jacobs and Simpson have this distinction,
  we do not think that this distinction is important for our use case
  \todo{is it?}
\item[Categorising traces]
  \langname{} is a verification language and (contravening UML2)
  uses a two-category approach: valid/other or invalid/other, depending on the
  \metafeature{isNegated} flag on the \massertion.
  As the main \langname{} semantics is \tockcsp, we use a
  refinement-style categorisation where a sequence denotes an
  expected set of traces (or stable failures, and so on) that must be a superset
  of those calculated from the target.
\item[Complete or partial traces]
  \langname{} sequences (contravening UML2) are \emph{partial} with, by default,
  any suffix permitted.  Prefixes may be permitted by using gaps at the start of
  the sequence.  This difference reflects the underlying trace model of \tockcsp.
  \todo{Ana mentioned a good reason for why we might want this semantics that
  isn't just `CSP does it', but I can't remember precisely what it was.
  Compositionality?}
\item[Combining fragments]
  \langname{} sequences (contravening UML2) currently synchronise on entering
  their equivalent of combined fragments.  \todo{This doesn't actually matter
  yet for 2-actor sequences.}
\item[Processing the diagram]
  \langname{} sequences tend to have a compositional semantics whereby metamodel
  concepts translate bottom-up into fragments of
  the object language, with combined fragments translating into the relevant
  operators,
  and additional superstructure such as memory processes and parameterisations
  being added as needed.
\item[Underlying formalism: Approach]
  The \langname{} semantics maps each sequence to a representation in another
  language, to which we delegate the choice of approach.
  In at least the example of \tockcsp, the end result is a calculation of the
  set of all possible traces (or similar denotation), which corresponds at
  least partially to UML2.
\item[Underlying formalism: Concurrency]
  \langname{} sequences have an interleaving semantics, in accordance with
  both UML2 and \tockcsp.
\item[Handling choices]
  Not set yet
  \todo{I suspect this will create a global time point for the choice.}
\item[Interpretation of a false guard]
  Not set yet.
  \todo{I suspect this will act as in CSP eg pulling the alternative down to
  \(\Stop\).}
\item[Who evaluates a guard]
  Not set yet.
  \todo{I suspect this would be the memory process, which would suggest a
  global evaluation.}
\item[Formal and actual Gates]
  These do not exist in \langname{} sequences.
\item[Interpretation of \texttt{neg(S)}]
  These do not exist in \langname{} sequences.
\item[Ignore/consider]
  These do not exist in \langname{} sequences.
\item[Conformance-related operators in complex diagrams]
  \langname{} sequences have no conformance-related operators at the diagram
  level; sequences always represent a set of possible traces (or similar
  denotation) to be compared against those observable from a model.
\item[Traces being both valid and invalid]
  Theoretically, all traces are either allowed or disallowed by the diagram,
  with the lack of features for characterising traces within the diagram itself
  serving as a restriction to enforce this.  It may not
  be decidable to verify that all traces exhibitable by a model are allowed by
  the diagram (or that one trace disallowed by the diagram is never observable
  in the model).
\end{description}
