%!TEX root=./robocert.tex

% Target language colouration
\newcommand{\tlang}[1]{\textcolor{gray}{\boxed{#1}}}
% Object language colouration (nested inside target language)
\newcommand{\olang}[1]{\boxed{\textcolor{black}{#1}}}

\newcommand{\tockcsp}{\emph{tock}-CSP}
\newcommand{\cspm}{CSP\(_\text{M}\)}

\newcommand{\defeq}{\mathbin{\overset{\text{def}}=}}
% CSP operators
\newcommand{\interrupt}{\mathbin{\triangle}}
\newcommand{\cspnsop}{\mathbin{\!:\!:\!}}
% CSP keywords/processes
\newcommand{\cspkw}[1]{\operatorname{\mathbf{#1}}}
\newcommand{\runproc}[1]{\cspkw{Run}\left(#1\right)}
\newcommand{\events}{\cspkw{Events}}

%
% Metasyntactic variables
%
\newcommand{\apkg}{P}
\newcommand{\acsp}{f}
\newcommand{\anarrow}{a}
\newcommand{\anevent}{e}
\newcommand{\agap}{g}
\newcommand{\amspec}{m}
\newcommand{\amsgset}{M}
\newcommand{\aumsgset}{\amsgset_{\universe}}
\newcommand{\anemsgset}{\amsgset_{e}}
\newcommand{\aloop}{l}
\newcommand{\anop}{o}
\newcommand{\aseq}{\sigma}
\newcommand{\asseq}{q}
\newcommand{\astep}{s}
\newcommand{\atarget}{t}
\newcommand{\aninst}{\phi}
\newcommand{\aworld}{w}
\newcommand{\anaction}{\alpha}
\newcommand{\anexpr}{x}
\newcommand{\avar}{v}
% Assertions
\newcommand{\anasst}{\alpha}
\newcommand{\asasst}{\alpha_s}
\newcommand{\amodel}{\mathcal{M}}

\newcommand{\sema}[2]{\llbracket #1 \rrbracket^{\mathsf{#2}}}
\newcommand{\pkgsema}[1]{\sema{#1}{pkg}}
\newcommand{\cspsema}[1]{\sema{#1}{csp}}
\newcommand{\stepsema}[1]{\sema{#1}{step}}
\newcommand{\exprsema}[2]{\sema{#1}{expr}_{(#2)}}
\newcommand{\gapsema}[2]{\sema{#1}{gap}_{(#2)}}
\newcommand{\actsema}[1]{\sema{#1}{act}}
\newcommand{\mspecsema}[1]{\sema{#1}{mspec}}
\newcommand{\loopsema}[1]{\sema{#1}{loop}}
\newcommand{\msgsetsema}[1]{\sema{#1}{mset}}
\newcommand{\seqsema}[1]{\sema{#1}{seq}}
\newcommand{\sseqsema}[1]{\sema{#1}{sseq}}
\newcommand{\asstsema}[1]{\sema{#1}{asst}}

\newcommand{\targetsema}[2]{\sema{#1}{target}_{(#2)}}

\newcommand{\ctargetRef}[1]{\funcname{ctargetRef}(#1)}
\newcommand{\otargetRef}[2]{\funcname{otargetRef}(#1, #2)}
\newcommand{\targetRef}[2]{\funcname{targetRef}(#1, #2)}

\newcommand{\funcname}[1]{\ensuremath{\mathsf{#1}}}
\newcommand{\eventsOf}[1]{\funcname{events}(#1)}
\newcommand{\seqnameOf}[1]{\funcname{seqName}(#1)}
\newcommand{\ctargetnameOf}[1]{\funcname{ctargetName}(#1)}
\newcommand{\otargetnameOf}[1]{\funcname{otargetName}(#1)}

\newcommand{\field}[2]{#1.\funcname{#2}}

This chapter formally captures the semantics of \langname{} in terms of its
target languages:

\begin{itemize}
\item
	\tockcsp~(\cref{sec:semantics-tockcsp});
\item
	\todo{PRISM};
\item
	\todo{Isabelle/UTP?}.
\end{itemize}

Each semantics captures \massertion s as the top-level definition, with all
objects reachable from the assertions translated in-line.  As a
consequence, we do not capture organisational details such as \mrapackage s,
or any distinction between references to objects and their definitions.

\section{How to read this section}

\begin{table}
	\centering

	\begin{tabular}{p{2em}p{11em}}
	\toprule
	\thead{Var.} & \thead{Type}
	\\
	\midrule
	\multicolumn{2}{l}{\tsubhead{Packages (\cref{sec:metamodel-top})}}
	\\
	\(\apkg\) & \mrapackage
	\\
	\(\acsp\) & \mcspfragment
	\\
	\midrule
	\multicolumn{2}{l}{\tsubhead{Sequences (\cref{sec:metamodel-sequences})}}
	\\
	\(\aseq\) & \msequence
	\\
	\(\asseq\) & \msubsequence
	\\
	\(\astep\) & \msequencestep
	\\
	\(\agap\) & \msequencegap
	\\
	\midrule
	\multicolumn{2}{l}{\tsubhead{Actions (\cref{sec:metamodel-actions})}}
	\\
	\(\anaction\) & \msequenceaction
	\\
	\(\anarrow\) & \marrowaction
	\\
	\(\aloop\) & \mloopaction
	\\
	\(\bot\) & \mfinalaction
	\\
	\\
	\bottomrule
	\end{tabular}
	\begin{tabular}{p{2em}p{11em}}
	\toprule
	\thead{Var.} & \thead{Type}
	\\
	\midrule
	\multicolumn{2}{l}{\tsubhead{Messages (\cref{sec:metamodel-messages})}}
	\\
	\(\amspec\) & \mmessagespec
	\\
	\(\amsgset\) & \mgapmessageset
	\\
	\(\aumsgset\) & \muniversegapmessageset
	\\
	\(\anemsgset\) & \mextensionalgapmessageset
	\\
	\midrule
	\multicolumn{2}{l}{\tsubhead{Actors (\cref{sec:metamodel-actors})}}
	\\
	\(\atarget\) & \mtarget
	\\
	\(\aworld\) & \mworld
	\\
	\(\aninst\) & \mtargetinstantiation
	\\
	\(\avar\) & \mvariable
	\\
	\(\anexpr\) & \mexpression
	\\
	\midrule
	\multicolumn{2}{l}{\tsubhead{Assertions (\cref{sec:metamodel-assertions})}}
	\\
	\(\anasst\) & \massertion
	\\
	\(\asasst\) & \msequenceassertion
	\\
	\(\amodel\) & \mcspmodel	
	\\
	\bottomrule
	\end{tabular}

	\caption{Metasyntactic variables.}
	\label{tab:metasyntactic-variables}
\end{table}

The semantics treatments in this section take the form of rewrite rules from
the abstract syntax in \cref{cha:metamodel} to some object language (for
instance, \tockcsp).
For conciseness, we use a meta-language based on the Z notation.
\todo{is this ok?  does it need better explanation?}
We also use the following notational conventions \todo{may need tweaking}:

\begin{itemize}
\item
	\tlang{\text{boxed shaded text}} denotes a construct from the object
	language; outside such text, or \tlang{\olang{\text{inside nested boxes}}},
	assume use of the meta-language;
\item
	\(\sema{-}{name}\) denotes a main semantic rule;
\item
	\(\funcname{name}()\) denotes auxiliary semantic functions;
\item
	\(\field{x}{name}\) denotes a field of the metamodel object \(x\);
\item
	variables in the meta-language have an implicit metamodel type
	corresponding to one of the metasyntactic variables in
	\cref{tab:metasyntactic-variables}.
\end{itemize}

\section{\tockcsp{} semantics}\label{sec:semantics-tockcsp}
%!TEX root=../robocert.tex

This section introduces a \tockcsp{} semantics for \langname.
\todo{Need to make sure it's actually \tockcsp, not CSP.}
This semantics is inspired by that of Lima et al. on the CML semantics of
UML sequence diagrams.

The behaviour of the \langname{} generator is expected to conform to
this semantics, except that:

\begin{itemize}
\item
  The generator targets \cspm{} instead of CSP, and so \emph{may}
  use semantically equivalent \cspm{} constructs where the CSP equivalents
  are missing or not idiomatic;
\item
  The generator \emph{must} \todo{not yet but eventually} wrap processes in
  timed sections and prioritisations to achieve the appropriate \tockcsp{} behaviours of
  CSP operators;
\item
  The generator \emph{may} perform semantics-preserving optimisations,
  such as substituting \(P\) for \(\tlang{\Stop \interrupt \olang{P}}\).
\end{itemize}

\todo{Some of the definitions are very loosely typed (as they are expanding from
a metamodel to textual snippets of something halfway between \tockcsp{} and
\cspm), and it'd be nice to fix that and/or give explicit result types.}

\subsection{Assertions}

The definitions here correspond to \cref{sec:metamodel-assertions}.

\begin{definition}[\massertion]

\newcommand{\refop}[3]{\mathbin{\odot_{#1}^{(#2, #3)}}}

The only type of assertion so far is \msequenceassertion.  The semantics of a
\msequenceassertion{} relates the \msequence{} of the assertion to the
\mtarget{} of the assertion.
%
\begin{align*}
	\asstsema{\asasst}
\quad\defeq\quad&
	\seqnameOf{\field{\asasst}{sequence}}
	\;
	\refop{\field{\asasst}{model}}{\field{\asasst}{isNegated}}{\field{\asasst}{type}}
	\;
	\targetsema{\field{\field{\asasst}{sequence}}{target}}{\field{\asasst}{instantiation}}
\end{align*}

The exact refinement operator depends on the assertion type and negation:
%
\begin{align*}
	\refop{\amodel}{\true}{\mathsf{holds}}
\quad\defeq\quad&
	\tlang{\sqsubseteq_{\olang{\amodel}}}
&
	\refop{\amodel}{\false}{\mathsf{holds}}
\quad\defeq\quad&
	\tlang{\not\sqsubseteq_{\olang{\amodel}}}
\\
	\refop{\amodel}{\true}{\mathsf{isObserved}}
\quad\defeq\quad&
	\tlang{\sqsupseteq_{\olang{\amodel}}}
&
	\refop{\amodel}{\false}{\mathsf{isObserved}}
\quad\defeq\quad&
	\tlang{\not\sqsupseteq_{\olang{\amodel}}}
\\
\end{align*}
\end{definition}


\subsection{Sequences}\label{ssec:semantics-tockcsp-sequences}

The definitions here correspond to \cref{sec:metamodel-sequences}.

\begin{definition}[\msequence]

The semantics of a \msequence{} is that of its subsequence
\todo{eventually, in parallel with its memory}.
%
\begin{align*}
	\seqsema{\aseq}
\quad\defeq\quad&	
	\sseqsema{\field{\aseq}{body}}
\end{align*}

\end{definition}

\begin{definition}[\msubsequence]

The semantics of a \msubsequence{} is a sequential composition of that of its steps.
%
\begin{align*}
	\sseqsema{\asseq}
	\quad\defeq\quad&	
	\funcname{steps}(\field{\asseq}{body})
\\
	\funcname{steps}(\langle \astep_1, \dotsc, \astep_n \rangle)
	\quad\defeq\quad&	
	\tlang{
	\olang{\stepsema{\astep_1}}
	\circseq
	\olang{\dotso}
	\circseq
	\olang{\stepsema{\astep_n}}
	}
\end{align*}

\end{definition}

\todo{\msequencestep}

\begin{definition}[\mactionstep]

The semantics of a \mactionstep{} is an unbounded loop over the events of its
\msequencegap, interrupted by the \msequenceaction.\footnote{Note that the semantics of the gap depends
on the action.  This is because, as seen in \(\gapsema{\agap}{\anaction}\),
the events on which the action can interrupt the
gap must be removed from the events of the gap to ensure the interrupt has the
expected semantics.}
%
\begin{align*}
	\stepsema{\astep}
\quad\defeq\quad&	
	\tlang{
		\runproc{\olang{\gapsema{\field{\astep}{gap}}{\field{\astep}{action}}}}
		\interrupt \olang{\actsema{\field{\astep}{action}}}
	}
\end{align*}
\end{definition}

\begin{definition}[\msequencegap]
	The semantics of a \msequencegap{} is the CSP event set corresponding to
	the difference between the \emph{allowed} set and any events
	the action following the gap can initially communicate.
%
\begin{align*}
	\gapsema{
		\agap
	}{\anaction}
\quad\defeq\quad&
\tlang{
	\olang{\msgsetsema{\field{\agap}{allowed}}}
	\setminus
	\olang{\eventsOf{\anaction}}
}
\end{align*}
\end{definition}

\begin{definition}[CSP event sets]
For now, the event set of an action is the prefix of an arrow action, or
the empty set for any other actions.  This will change when the prefix becomes
inexpressible as an event set.
%
\begin{align*}
	\eventsOf{\anarrow}
\quad\defeq\quad&
	\tlang{\Set{\olang{\emspecsema{\anarrow}}}}
	\tag{arrow}
\\
	\eventsOf{\anaction}
\quad\defeq\quad&
	\tlang{\emptyset}
	\tag{anything else}
\end{align*}
\end{definition}

\subsection{Actions}\label{ssec:semantics-tockcsp-actions}

The definitions here correspond to \cref{sec:metamodel-actions}.

\begin{definition}[\msequenceaction]

The semantics of an arrow action is the semantics of its arrow as a CSP prefix,
prefixing termination.  The semantics of a loop delegates to a separate rule
over its label and subsequence.
%
\begin{align*}
	\actsema{\anarrow}
\quad\defeq\quad&
	\tlang{
	\left(
	\olang{\pmspecsema{\anarrow}}
	\then
	\Skip
	\right)
	}
	\tag{arrow action}
\\
	\actsema{\aloop}
\quad\defeq\quad&
	\loopsema{\aloop}
\tag{loop action}
\\
	\actsema{\bot}
\quad\defeq\quad&
	\tlang{\Skip}
\tag{final action}
\end{align*}

\end{definition}

\newcommand{\iloop}[1]{\text{Loop}(#1)}
\newcommand{\nloop}[1]{\text{Loop}_\sqcap(#1)}
\newcommand{\dloop}[2]{\text{Loop}_d(#1, #2)}
\newcommand{\lloop}[2]{\text{Loop}_l(#1, #2)}
\newcommand{\uloop}[2]{\text{Loop}_u(#1, #2)}
\newcommand{\rloop}[3]{\text{Loop}_r(#1, #2, #3)}

\begin{definition}[\mloopstep]  
For now, the semantics of a loop is that of the loop
subsequence hoisted into an auxiliary loop-running process (defined
below).
\todo{define the symbols used here}
%
\begin{align*}
  \loopsema{\aloop}
  \quad\defeq\quad
  &
    \funcname{runner}(\field{\aloop}{bound}, \sseqsema{\field{\aloop}{body}})
  \\
  \funcname{runner}(\infty, P)
  \quad\defeq\quad
  & \tlang{\iloop{\olang{P}}}
    \tag{infinite loops}
  \\
  \funcname{runner}(n, P)
  \quad\defeq\quad
  & \tlang{\dloop{\olang{\exprsema{n}{\emptyset}}}{\olang{P}}}
    \tag{definite-bound loops}
  \\
  \funcname{runner}((l, \ast), P)
  \quad\defeq\quad
  & \tlang{\lloop{\olang{\exprsema{l}{\emptyset}}}{\olang{P}}}
    \tag{lower-bound loops}
  \\
  \funcname{runner}((l, u), P)
  \quad\defeq\quad
  & \tlang{\rloop{\olang{\exprsema{l}{\emptyset}}}{\olang{\exprsema{u}{\emptyset}}}{\olang{P}}}
    \tag{range-bound loops}
  \\  
\end{align*}

\end{definition}

\begin{definition}[Auxiliary loop processes]

We can define the loop processes used above in CSP as follows:
%
\begin{align*}
  \iloop{P}
  \quad\defeq\quad
  & P \circseq \iloop{P}
    \tag{infinite}
  \\
  \nloop{P}
  \quad\defeq\quad
  & \Skip \sqcap \left(P \circseq \nloop{P}\right)
    \tag{nondeterministic}
  \\  
  \dloop{n}{P}
  \quad\defeq\quad
  & \Skip \triangleleft n \leq 0 \triangleright \left(P \circseq \dloop{n-1}{P}\right)
    \tag{definite}
  \\
  \uloop{n}{P}
  \quad\defeq\quad
  & \Skip \triangleleft n \leq 0 \triangleright \left(\Skip \sqcap \left(P \circseq \uloop{n-1}{P}\right)\right)
    \tag{upper-bound}
  \\
  \lloop{n}{P}
  \quad\defeq\quad
  & \dloop{n}{P} \circseq \nloop{P}
    \tag{lower-bound}
  \\
  \rloop{l}{u}{P}
  \quad\defeq\quad
  & \lloop{l}{P} \circseq \uloop{u - l}{P}
    \tag{range}
  \\
\end{align*}
\end{definition}

\subsection{Messages}\label{ssec:semantics-tockcsp-messages}

The definitions here correspond to \cref{sec:metamodel-messages}.

\begin{definition}[\mmessageset]

  The semantics of a message set is the universal event set for \muniversemessageset s;
  the expansion of the given \mgapmessagespec s for \mextensionalmessageset s;
  the semantics of the referred-to set for \mrefmessageset s;
  and the semantics of the referred-to set operator for \mbinarymessageset s.
%
\begin{align*}
  \msgsetsema{\aumsgset}
  \quad\defeq\quad
  &
    \tlang{\events}
    \tag{universe}
  \\
  \intertext{\todo{The meta/object language distinction is messy here
  and I'm not sure how to handle it.}}
  \msgsetsema{\anemsgset}
  \quad\defeq\quad
  &
    \tlang{
    \bigcup
    \olang{
    \Set{
    \emspecsema{\amspec} | \amspec \in \field{\anemsgset}{messages}
    }
    }
    }
    \tag{extensional}
  \\
  \msgsetsema{\armsgset}
  \quad\defeq\quad
  &
    \msgsetsema{\field{\field{\armsgset}{set}}{set}}
    \tag{reference}
  \\
  \msgsetsema{\amsgset_l \odot \amsgset_r}
  \quad\defeq\quad
  &
  \tlang{
    \olang{\msgsetsema{\amsgset_l}}
    \odot
    \olang{\msgsetsema{\amsgset_r}}
    \tag{binary-operator}
  }
\end{align*}
\end{definition}

There are two distinct semantic rules for \mmessagespec s.  One is used whenever
we expand the \mmessagespec{} into an event set, and handles `rest' arguments
by implicitly quantifying over the missing parameters.  The other is used when
expanding to a prefix, and introduces explicit discards for the missing
parameters.  \todo{The two forms will diverge further when we introduce
parameter binding.}

\begin{definition}[\mmessagespec{} as an event set]

We expand \mmessagespec s into event-set form in two stages: expanding
any non-`rest' arguments with \funcname{withArgs}, then expanding the channel
reference using \funcname{msgChan}.
%
\begin{align*}
	\emspecsema{\amspec}
\quad\defeq\quad&
\funcname{withArgs}\left(\field{\amspec}{arguments}, \amspec\right)
\\
	\funcname{withArgs}\left(\amspec, \anarglist\cat\langle\anexprarg\rangle\right)
\quad\defeq\quad&
\tlang{
	\olang{\funcname{withArgs}\left(\amspec, \anarglist\right)}
	!\olang{\exprsema{\anexprarg}{\amspec}}
}
\tag{expression argument}
\\
	\funcname{withArgs}\left(\amspec, \anarglist\cat\langle\arestarg\rangle\right)
\quad\defeq\quad&
	\funcname{withArgs}\left(\amspec, \anarglist\right)
\tag{rest argument}
\\
	\funcname{withArgs}\left(\amspec, \langle\rangle\right)
\quad\defeq\quad&
	\funcname{msgChan}\left(\amspec\right)
\tag{end of arguments}
\end{align*}
The expansion of the message channel
depends on the message topic
and, for certain topics, the direction (inbound from world to target, or
outbound from target to world).
\newcommand{\nsOf}[1]{\mathsf{ns}(#1)}
\newcommand{\targetOf}[2]{\mathsf{target}(#1,#2)}
\newcommand{\topicOf}[3]{\mathsf{topic}(#1,#2,#3)}
%
\begin{align*}
	\funcname{msgChan}(\amspec)
\quad\defeq\quad&
\tlang{
	\olang{\nsOf{\targetOf{\field{\amspec}{from}}{\field{\amspec}{to}}}}
	{}\cspnsop{}
	\olang{\topicOf{\field{\amspec}{topic}}{\field{\amspec}{from}}{\field{\amspec}{to}}}
}
\\
	\targetOf{\aworld}{\atarget}
\quad\defeq\quad&
	\targetOf{\atarget}{\aworld}
	\quad\defeq\quad
	\atarget
\\
	\topicOf{\anop}{\atarget}{\aworld}
\quad\defeq\quad&
\tlang{
	\olang{\field{\anop}{name}}\text{Call}
}
\tag{operation; always outbound}
\\
	\topicOf{\anevent}{\aworld}{\atarget}
\quad\defeq\quad&
	\tlang{\olang{\field{\anevent}{name}}\text{.in}}
\tag{inbound event}
\\
	\topicOf{\anevent}{\atarget}{\aworld}
\quad\defeq\quad&
	\tlang{\olang{\field{\anevent}{name}}\text{.out}}
\tag{outbound event}
\end{align*}
\end{definition}

\newcommand{\paramsOf}[1]{\funcname{params}(#1)}
\newcommand{\beforeRest}[1]{\funcname{beforeRest}(#1)}

\begin{definition}[\mmessagespec{} as a prefix]

To expand a \mmessagespec{} into a prefix, we take the event set expansion and
\todo{for now} \emph{pad} it with wildcard inputs for each topic parameter
left unmatched at the point of any \mrestargument.
%
\begin{align*}
	\pmspecsema{\amspec}
\quad\defeq\quad&
	\funcname{pad}\left(\emspecsema{\amspec}, \funcname{padding}(\amspec)\right)
\\
	\funcname{pad}(x, 0)
\quad\defeq\quad&
	x
\tag{base case}
\\
	\funcname{pad}(x, n)
\quad\defeq\quad&
\tlang{
	\olang{\funcname{pad}(x, n-1)}?\_
}
\tag{inductive case}
\\
	\funcname{padding}(\amspec)
\quad\defeq\quad&
	\#(\paramsOf{\field{\amspec}{topic}}) - \beforeRest{\field{\amspec}{arguments}}
\\
	\beforeRest{\langle\rangle}
\quad\defeq\quad&
	0
\tag{base case}
\\
	\beforeRest{\langle \arestarg \rangle \cat \anarglist}
\quad\defeq\quad&
	0
\tag{rest argument}
\\
	\beforeRest{\langle \anarg \rangle \cat \anarglist}
\quad\defeq\quad&
	1 + \beforeRest{\anarglist}
\tag{non-rest argument}
\end{align*}
\end{definition}

\subsection{Actors}\label{ssec:semantics-tockcsp-actors}

The definitions here correspond to \cref{sec:metamodel-actors}.


\begin{definition}[\mtarget]

The semantics of a target is the parametric process generated for that
target by the relevant external semantics, with constant parameters instantiated
as follows:

\begin{itemize}
\item
	if the constant is bound in the assertion's \mtargetinstantiation{}
	(passed to the semantics as \(\aninst\)), use the bound expression; else
\item
	if the constant is bound in the target's \mtargetinstantiation, use
	the bound expression; else
\item
	use the \(\funcname{constName}\) of the constant, under the assumption
	that it will later bind to a fallback instantiation for the constant.
\end{itemize}
%
\begin{align*}
	\targetsema{\atarget}{\aninst}
\quad\defeq\quad&
\tlang{
	\olang{\funcname{targetProcess}(\atarget)}
	\left(
		\olang{\exprsema{-}{\atarget}\ \circ\ \funcname{instantiate}(\aninst)\ \circ\ \funcname{targetParams}(\atarget)}
	\right)
}
\\
	\funcname{instantiate}(\aninst, \atarget)
\quad\defeq\quad&
	\funcname{constName}
	\oplus
	\field{\field{\atarget}{instantiation}}{constants}
	\oplus
	\field{\aninst}{constants}
\end{align*}
\end{definition}


%%% Local Variables:
%%% mode: latex
%%% TeX-master: "../robocert"
%%% End:

