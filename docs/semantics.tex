%!TEX root=./robocert.tex

% Target language colouration
\newcommand{\tlang}[1]{\textcolor{blue}{#1}}
% Object language colouration (nested inside target language)
\newcommand{\olang}[1]{\textcolor{black}{#1}}

\newcommand{\tockcsp}{\emph{tock}-CSP}
\newcommand{\cspm}{CSP\(_\text{M}\)}

\newcommand{\defeq}{\mathbin{\overset{\text{def}}=}}
% CSP operators
\newcommand{\interrupt}{\mathbin{\triangle}}
\newcommand{\cspnsop}{\mathbin{\!:\!:\!}}
% CSP keywords/processes
\newcommand{\cspkw}[1]{\operatorname{\mathbf{#1}}}
\newcommand{\runproc}[1]{\cspkw{Run}\left(#1\right)}
\newcommand{\events}{\cspkw{Events}}

%
% Metasyntactic variables
%
\newcommand{\apkg}{P}
\newcommand{\acsp}{f}
\newcommand{\anarrow}{a}
\newcommand{\anevent}{e}
\newcommand{\agap}{g}
\newcommand{\amspec}{m}
\newcommand{\amsgset}{M}
\newcommand{\aumsgset}{\amsgset_{\universe}}
\newcommand{\anemsgset}{\amsgset_{e}}
\newcommand{\aloop}{l}
\newcommand{\anop}{o}
\newcommand{\aseq}{\sigma}
\newcommand{\asseq}{q}
\newcommand{\astep}{s}
\newcommand{\atarget}{t}
\newcommand{\aninst}{\phi}
\newcommand{\aworld}{w}
\newcommand{\anaction}{x}
\newcommand{\avar}{v}
% Assertions
\newcommand{\anasst}{\alpha}
\newcommand{\asasst}{\alpha_s}
\newcommand{\amodel}{\mathcal{M}}

\newcommand{\sema}[2]{\llbracket #1 \rrbracket^{\mathsf{#2}}}
\newcommand{\pkgsema}[1]{\sema{#1}{pkg}}
\newcommand{\cspsema}[1]{\sema{#1}{csp}}
\newcommand{\stepsema}[1]{\sema{#1}{step}}
\newcommand{\gapsema}[2]{\sema{#1}{gap}_{(#2)}}
\newcommand{\actsema}[1]{\sema{#1}{act}}
\newcommand{\mspecsema}[1]{\sema{#1}{mspec}}
\newcommand{\loopsema}[1]{\sema{#1}{loop}}
\newcommand{\msgsetsema}[1]{\sema{#1}{mset}}
\newcommand{\seqsema}[1]{\sema{#1}{seq}}
\newcommand{\sseqsema}[1]{\sema{#1}{sseq}}
\newcommand{\asstsema}[1]{\sema{#1}{asst}}

\newcommand{\ctargetsema}[1]{\sema{#1}{ctarget}}
\newcommand{\otargetsema}[1]{\sema{#1}{otarget}}

\newcommand{\ctargetRef}[1]{\funcname{ctargetRef}(#1)}
\newcommand{\otargetRef}[2]{\funcname{otargetRef}(#1, #2)}
\newcommand{\targetRef}[2]{\funcname{targetRef}(#1, #2)}

\newcommand{\funcname}[1]{\ensuremath{\mathsf{#1}}}
\newcommand{\eventsOf}[1]{\funcname{events}(#1)}
\newcommand{\seqnameOf}[1]{\funcname{seqName}(#1)}
\newcommand{\ctargetnameOf}[1]{\funcname{ctargetName}(#1)}
\newcommand{\otargetnameOf}[1]{\funcname{otargetName}(#1)}

\newcommand{\field}[2]{#1.\funcname{#2}}

This chapter formally captures the semantics of \langname{} in terms of its
target languages:

\begin{itemize}
\item
	\tockcsp~(\cref{sec:semantics-tockcsp});
\item
	\todo{PRISM};
\item
	\todo{Isabelle/UTP?}.
\end{itemize}

\section{How to read this section}

\begin{table}
	\centering

	\begin{tabular}{p{2em}p{11em}}
	\toprule
	\thead{Var.} & \thead{Meaning}
	\\
	\midrule
	\multicolumn{2}{l}{\tsubhead{Packages (\cref{sec:metamodel-top})}}
	\\
	\(\apkg\) & \mrapackage
	\\
	\(\acsp\) & \mcspfragment
	\\
	\midrule
	\multicolumn{2}{l}{\tsubhead{Sequences (\cref{sec:metamodel-sequences})}}
	\\
	\(\aseq\) & \msequence
	\\
	\(\asseq\) & \msubsequence
	\\
	\(\astep\) & \msequencestep
	\\
	\(\agap\) & \msequencegap
	\\
	\midrule
	\multicolumn{2}{l}{\tsubhead{Actions (\cref{sec:metamodel-actions})}}
	\\
	\(\anaction\) & \msequenceaction
	\\
	\(\anarrow\) & \marrowaction
	\\
	\(\aloop\) & \mloopaction
	\\
	\(\bot\) & \mfinalaction
	\\
	\\
	\bottomrule
	\end{tabular}
	\begin{tabular}{p{2em}p{11em}}
	\toprule
	\thead{Var.} & \thead{Meaning}
	\\
	\midrule
	\multicolumn{2}{l}{\tsubhead{Messages (\cref{sec:metamodel-messages})}}
	\\
	\(\amspec\) & \mmessagespec
	\\
	\(\amsgset\) & \mgapmessageset
	\\
	\(\aumsgset\) & \muniversegapmessageset
	\\
	\(\anemsgset\) & \mextensionalgapmessageset
	\\
	\midrule
	\multicolumn{2}{l}{\tsubhead{Actors (\cref{sec:metamodel-actors})}}
	\\
	\(\atarget\) & \mtarget
	\\
	\(\aworld\) & \mworld
	\\
	\(\aninst\) & \mtargetinstantiation
	\\
	\(\avar\) & \mvariable
	\\
	\midrule
	\multicolumn{2}{l}{\tsubhead{Assertions (\cref{sec:metamodel-assertions})}}
	\\
	\(\anasst\) & \massertion
	\\
	\(\asasst\) & \msequenceassertion
	\\
	\(\amodel\) & \mcspmodel	
	\\
	\bottomrule
	\end{tabular}

	\caption{Metasyntactic variables.}
	\label{tab:metasyntactic-variables}
\end{table}

The semantics treatments in this section take the form of rewrite rules from
the abstract syntax in \cref{cha:metamodel} to some object language (for
instance, \tockcsp).
For conciseness, we use a meta-language based on the Z notation.
\todo{is this ok?  does it need better explanation?}
We also use the following notational conventions \todo{may need tweaking}:

\begin{itemize}
\item
	\tlang{blue text} denotes a construct from the object
	language; outside such text, assume use of the meta-language;
\item
	\(\sema{-}{name}\) denotes a main semantic rule;
\item
	\(\funcname{name}()\) denotes auxiliary semantic functions;
\item
	\(\field{x}{name}\) denotes a field of the metamodel object \(x\);
\item
	variables in the meta-language have an implicit metamodel type
	corresponding to one of the metasyntactic variables in
	\cref{tab:metasyntactic-variables}.
\end{itemize}

\section{\tockcsp{} semantics}\label{sec:semantics-tockcsp}
%!TEX root=../robocert.tex

This section introduces a \tockcsp{} semantics for \langname.

\section{Core language}\label{sec:semantics-tockcsp-core}

\section{Sequence notation}\label{sec:semantics-tockcsp-seq}
%!TEX root=robocert.tex

The \emph{sequence} notation of \langname{} provides a method of
defining the expected interactions between actors in a robotic model,
optionally with time constraints.  For instance, sequences may specify
how a \robochart{} module uses services offered by a robotic platform.
Sequences resemble UML sequence diagrams.

\chapter{Metamodel}\label{cha:metamodel}
%!TEX root=../robocert.tex
\input{seq/metamodel/defs}

This chapter discusses the metamodel of \langname{} sequences.  This metamodel
has multiple concrete notations ---
the \emph{textual} (\cref{cha:seq-textual}) and \emph{graphical}
syntax (\cref{cha:seq-graphical}) --- and a semantics (\cref{cha:semantics-intro}). 

\section{Introduction}\label{sec:metamodel-intro}

For information on how to read the rest of this chapter, see the notes on
the top-level metamodel (\cref{ssec:core-metamodel-intro-readme}).

\subsection{A note on ordering and timing}\label{ssec:metamodel-intro-ordering}

By default, \langname{} sequences are \emph{explicit}
as to \emph{which} actions occur, but \emph{implicit} as to
\emph{when}.

Unless modified by gaps,
communications on \langname{} sequences are strictly ordered with
respect to each other and permit no intervening communications.  This
reflects the view of sequence diagrams as representations of traces of
the system under test, and appears consistent with some semantic
treatments of UML~\cite{lima-semantics}.  \todo{needs more
  justification?  is the point about Lima's semantics correct?}

Like \robochart, \langname{} has a discrete-time model, where time is
measured in \emph{time units} and events, operation calls, and
primitive data operations (assignments, communication, and so on) are
instantaneous.  Time cn pass when the software is waiting for an
interaction with the platform.  Accordingly, any amount of time may
pass in a sequence, unless constrained by \mdeadlinestep s
(\cref{ssec:metamodel-steps-deadlines}).

\section{Sequences}\label{sec:metamodel-sequences}
\input{seq/metamodel/sequences}

\section{Steps}\label{sec:metamodel-steps}
\input{seq/metamodel/steps}

\section{Actions}\label{sec:metamodel-actions}
\input{seq/metamodel/actions}

\section{Messages}\label{sec:metamodel-messages}
\input{seq/metamodel/messages}

\section{Actors}\label{sec:metamodel-actors}
\input{seq/metamodel/actors}

\section{Assertions}\label{sec:seq-metamodel-assertions}
\input{seq/metamodel/assertions}

%%% Local Variables:
%%% mode: latex
%%% TeX-master: "../robocert"
%%% End:


\chapter{Textual syntax}\label{cha:textual}
This section describes the textual syntax of \langname.

\todo{TODO}

%%% Local Variables:
%%% mode: latex
%%% TeX-master: "robocert"
%%% End:


\chapter{Graphical syntax}\label{cha:graphical}
%!TEX root=../robocert.tex

This section describes the graphical syntax of \langname.

\todo{TODO}

\section{Principles}

\todo{These need work.}

\begin{itemize}
\item Where possible, we draw graphical cues from existing notations.
  This simplifies the uptake of the language for practitioners
  familiar with those notations.  Generally, cues for features come
  from whichever language is identified as the main inspiration in
  \cref{sec:seq-comparison-features}.
\item Where graphical elements attach to one side of a sequence diagram,
  we choose the side that is most relevant to that element.  For instance:
  \begin{itemize}
  \item Gaps on \marrowaction s attach to whichever
    side is the source of the arrow, to emphasise that the gap modifies
    the act of \emph{sending} the message \todo{I'm not sure this is convincing};
  \item Gaps on \mfinalaction s attach to the \mtarget{} end,
    since the \mfinalaction{} is similar to a message from the \mtarget{} to
    the \mworld{} specifying the \mtarget{} is finished.
  \end{itemize}
\end{itemize}

%%% Local Variables:
%%% mode: latex
%%% TeX-master: "robocert"
%%% End:

%%% Local Variables:
%%% mode: latex
%%% TeX-master: "robocert"
%%% End:


%%% Local Variables:
%%% mode: latex
%%% TeX-master: "../robocert"
%%% End:
