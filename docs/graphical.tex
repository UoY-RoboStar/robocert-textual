This section describes the graphical syntax of \langname.

\todo{TODO}

\section{Principles}

\todo{These need work.}

\begin{itemize}
\item Where possible, we draw graphical cues from existing notations.
  This simplifies the uptake of the language for practitioners
  familiar with those notations.  Generally, cues for features come
  from whichever language is identified as the main inspiration in
  \cref{ssec:metamodel-intro-features}.
\item Where graphical elements attach to one side of a sequence diagram,
  we choose the side that is most relevant to that element.  For instance:
  \begin{itemize}
  \item \msequencegap s on \marrowaction s attach to whichever
    side is the source of the arrow, to emphasise that the gap modifies
    the act of \emph{sending} the message \todo{I'm not sure this is convincing};
  \item \msequencegap s on \mfinalaction s attach to the \mtarget{} end,
    since the \mfinalaction{} is similar to a message from the \mtarget{} to
    the \mworld{} specifying the \mtarget{} is finished.
  \end{itemize}
\end{itemize}

%%% Local Variables:
%%% mode: latex
%%% TeX-master: "robocert"
%%% End: