%!TEX root=../robocert.tex

This chapter discusses the metamodel and notation available for including
\cspm{} fragments into \langname{} developments.

\section{Metamodel}

\subsection{Marker classes}

\todo{Is there a better way of doing this?  This seems bad.}

There are two abstract classes implemented by various parts of the \langname{}
metamodel that mark those parts as being available for use in aspects of
low-level CSP developments:

\begin{itemize}
\item
	\mcspprocesssource{} denotes objects that produce CSP processes;
\item
	\mcspcontextsource{} denotes objects that produce \emph{contexts}
	(bounds on the event set that must be considered when implementing the
	\emph{tick-tock} semantic model by use of model shifting).
\end{itemize}

\subsection{\mcspgroup}

A \mcspgroup{} contains zero or more \mcspfragment s.

By default, the generator \rfcshould{} insert the
\mcspgroup 's contents into the generated \cspm{} script at the
location implied by the group's ordering in the \mrapackage.  In other words,
the \mcspgroup{} \rfcshould{} have access to any \cspm{} generated for any
preceding elements.

If the \metafeature{preamble} feature is set, the generator \rfcmust{} instead
insert the \mcspgroup{} \emph{before} generating any imports (but \rfcshould{}
still preserve its order relative to other \metafeature{preamble} \mcspgroup s.
This is useful mainly for debugging purposes.

\subsection{\mcspfragment}

A \mcspfragment{} contains an unprocessed fragment of \cspm{} (the
\metafeature{body}), to be inserted
into any \cspm{} generated from a \mrapackage.
There are three types of \mcspfragment, discussed below.
\todo{rename features in metamodel to match}

\paragraph{\minlinecspfragment}

A \minlinecspfragment{} is a nameless fragment whose \metafeature{body} is
spliced directly into the generated \cspm{} script.  Such fragments are useful
for debugging, temporarily fixing or adjusting aspects of the \cspm{}
generation, and performing tasks that are not yet captured by \langname.

\begin{remark}
As \metafeature{body} is not processed, nothing defined inside will be
visible for reference in the \mrapackage.
\end{remark}

\begin{lstlisting}[style=Example]
inline <$ print "Hello, world" $>
\end{lstlisting}

\paragraph{\meventsetcspfragment}

\todo{I'm not convinced this should exist, or, at least, it should be renamed
or split up.}

A \meventsetcspfragment{} binds a
\metafeature{body} (representing an event set) to a
\metafeature{name}, such that it can be used for defining
the \emph{tick-tock} model-shifting context for a
\mprocesscspfragment.  Each \meventsetcspfragment{} is a \mcspcontextsource.

\begin{remark}
Since the main purpose of a \meventsetcspfragment{} is to set up a
model-shifting context, generators \rfcmay{} output additional definitions into
the top-level module for the \mrapackage{} on encountering a
\meventsetcspfragment.
\end{remark}

\begin{lstlisting}[style=Example]
event set E <$ Segway::sem__events $>
\end{lstlisting}

\paragraph{\mprocesscspfragment}

A \mprocesscspfragment{} binds a \metafeature{body} (representing a process)
to a \metafeature{name}, such that it becomes available in
\mcsprefinementproperty{} operands.  Each \mprocesscspfragment{} is a
\mcspprocesssource.

\begin{lstlisting}[style=Example]
process P <$ STOP $>

// for tick-tock reasoning, we'll need to define a context:
event set E <$ Segway::sem__events $>
process P: events in E <$ STOP $>
\end{lstlisting}

\subsection{\mcsprefinementproperty}

A \mcsprefinementproperty{} specifies a \cspm{} refinement assertion.
Its \metafeature{lhs} and \metafeature{rhs} features
specify respectively the left and right hand sides of the \(\sqsubseteq\)
operation, as \mcspprocesssource s (so, for instance, a )

\begin{itemize}
\item
	If \metafeature{op} is \texttt{REFINES}, one \(\sqsubseteq\) will be
	generated; if it is \texttt{EQUALS}, a further inverted \(\sqsubseteq\)
	will also be generated to capture an equality relation.
\item
	If \metafeature{model} is \texttt{TRACES}, the refinement(s) will be
	checked under the traces model with maximal progress prioritisation;
	if it is \texttt{TICK\_TOCK}, they will be checked under 
\end{itemize}

\begin{remark}
A \texttt{TICK\_TOCK} refinement between two \mcspprocesssource s with different
underlying model-shifting contexts is ill-defined.  This may be explicitly
forbidden or automatically addressed in the future.
\end{remark}

\begin{lstlisting}[style=Example]
assertion A1: P refines SequenceGroup::Seq in the traces model

// P will need to use SequenceGroup's Target as a context for this to be well-formed
assertion A2: P refines SequenceGroup::Seq in the tick-tock model
\end{lstlisting}

%%% Local Variables:
%%% mode: latex
%%% TeX-master: "../robocert"
%%% End: