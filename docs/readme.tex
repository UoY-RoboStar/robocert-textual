%!TEX root=./robocert.tex

\emph{This manual is not yet aimed at an external audience}.

\section*{Structure}
This manual is laid out as follows:

\begin{enumerate}
\item
  Information about \langname, its purpose, and this manual;
\item
  An introduction to each notation supported by \langname, discussing the
  metamodel and any concrete notations attached to each;
\item
  The formal semantics of \langname, provided as a separate development for
  each target verification language (CSP, PRISM, etc.).
\end{enumerate}

\section*{Typography}
This report uses various typographical conventions:

\begin{itemize}
\item
	\todo{This style} represents open questions; these should be
	resolved before the report is closed.
\item
	\metaref{This style} represents metamodel class names.
\item
	\metafeature{This style} represents metamodel feature names.
\item
	\texttt{This style} represents concrete textual syntax.  Boldface
	denotes keywords in the textual language.
\end{itemize}

Certain chapters may have extra conventions, which will be documented in
similar sections at the start of the corresponding chapter.

The key words \rfcmust, \rfcmustnot, \rfcrequired, \rfcshall, \rfcshallnot,
\rfcshould, \rfcshouldnot, \rfcrecommended, \rfcmay, and \rfcoptional{} in this
document are to be interpreted as described in RFC 2119.

%%% Local Variables:
%%% mode: latex
%%% TeX-master: "robocert"
%%% End:
