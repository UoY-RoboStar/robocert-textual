%!TEX root=./robocert.tex

\emph{This report is not yet aimed at an external audience}.

\paragraph{Typography}
This report uses various typographical conventions:

\begin{itemize}
\item
	\todo{This style} represents open questions; these should be
	resolved before the report is closed.
\item
	\metaref{This style} represents metamodel names.
\item
	\texttt{This style} represents concrete textual syntax.  Boldface
	denotes keywords in the textual language.
\end{itemize}

Certain chapters may have extra conventions, which will be documented in
similar sections at the start of the corresponding chapter.

The key words \rfcmust, \rfcmustnot, \rfcrequired, \rfcshall, \rfcshallnot,
\rfcshould, \rfcshouldnot, \rfcrecommended, \rfcmay, and \rfcoptional{} in this
document are to be interpreted as described in RFC 2119.

\paragraph{Known issues}
Aside from \todo{TODO comments}, this report has the following issues:

\begin{itemize}
\item
	Metamodel diagrams are of poor quality, partly due to working around
	\url{https://bugs.eclipse.org/bugs/show_bug.cgi?id=312723}.
\item
	As of writing, the semantics is incomplete.
\item
	There are no citations yet.	
\item
	Formatting is not aligned with the RoboStar reference manual house
	style.
\item
	There is no good distinction between the target and object languages
	in the semantics (there is a vague convention between serif and
	sans-serif fonts but this breaks down at symbols and brackets),
	which may be confusing.
\end{itemize}

\paragraph{Abbreviations} We make use of the following abbreviations:

\begin{description}
	\item[PSC] Property Sequence Chart~\cite{psc}
\end{description}
